
\documentclass[11pt]{article}

%Menge: \mathbb{R}

\usepackage[ngerman]{babel}
\usepackage{amsmath} %align, für = untereinander einfach &=
\usepackage{amssymb}
\usepackage{amsthm}
\usepackage{listings}
\usepackage{dsfont}
\usepackage[utf8]{inputenc}
\usepackage{graphicx}
\usepackage{esvect}
\graphicspath{ {./images/} }
\usepackage{tikz}         % For arrow and dots in \xvec
\usepackage[left=2.30cm, right=2.30cm, top=1.70cm, bottom=2.00cm]{geometry}

% --- Macro \xvec
\makeatletter
\newlength\xvec@height%
\newlength\xvec@depth%
\newlength\xvec@width%
\newcommand{\xvec}[2][]{%
	\ifmmode%
	\settoheight{\xvec@height}{$#2$}%
	\settodepth{\xvec@depth}{$#2$}%
	\settowidth{\xvec@width}{$#2$}%
	\else%
	\settoheight{\xvec@height}{#2}%
	\settodepth{\xvec@depth}{#2}%
	\settowidth{\xvec@width}{#2}%
	\fi%
	\def\xvec@arg{#1}%
	\def\xvec@dd{:}%
	\def\xvec@d{.}%
	\raisebox{.2ex}{\raisebox{\xvec@height}{\rlap{%
				\kern.05em%  (Because left edge of drawing is at .05em)
				\begin{tikzpicture}[scale=1]
				\pgfsetroundcap
				\draw (.05em,0)--(\xvec@width-.05em,0);
				\draw (\xvec@width-.05em,0)--(\xvec@width-.15em, .075em);
				\draw (\xvec@width-.05em,0)--(\xvec@width-.15em,-.075em);
				\ifx\xvec@arg\xvec@d%
				\fill(\xvec@width*.45,.5ex) circle (.5pt);%
				\else\ifx\xvec@arg\xvec@dd%
				\fill(\xvec@width*.30,.5ex) circle (.5pt);%
				\fill(\xvec@width*.65,.5ex) circle (.5pt);%
				\fi\fi%
				\end{tikzpicture}%
	}}}%
	#2%
}
\makeatother

\usepackage{blindtext}
%\usepackage{bookman}

% --- Override \vec with an invocation of \xvec.
\let\stdvec\vec
\renewcommand{\vec}[1]{\xvec[]{#1}}
% --- Define \dvec and \ddvec for dotted and double-dotted vectors.
\newcommand{\dvec}[1]{\xvec[.]{#1}}
\newcommand{\ddvec}[1]{\xvec[:]{#1}}

%eigene Befehle:
\newcommand{\R}{\mathbb{R}}
\newcommand{\nablavec}{\vec{\nabla}}

\title{Theoretische Physik 1}

\date{\today}
\begin{document}
\lstset{language=Java}
\author{Tom Herrmann}
\linespread{1.55}
\maketitle
 
\tableofcontents
\newpage

\section{Einleitung}
	\textbf{Gliederung:}
	\begin{itemize}
		\item \textbf{Mathematische Grundlagen:}\\Vektoren, krummlinige Koordinatensysteme, Differential- und Integralrechnung, partielle Ableitungen, Vektoranalysis, Gradient, Diverenz, Rotation, lineare gewöhnliche Differentialgleichungen, Matrizen und Tensoren, Fourier-Transformation, komplexe Zahlen, Wahrscheinlichkeit.
		\item \textbf{Grundlagen der Mechanik:} Kinematik und Dynamik von Massenpunktsystemen, Newtonsche Axiome, Arbeit, konservative Kräfte, Schwingungen, Zentralfeld, Kepler-Problem, beschleunigte Bezugssysteme
		\item \textbf{Spezielle Relativitätstheorie:}  Lorentz-Transformation und kinematische Konsequenzen, Minkowski-Raum, Vierer-Impuls, relativistische Bewegungsgleichung, Energie-Impuls-Vektor, Äquivalenz von Masse und Energie
		\item \textbf{Wärmelehre: } Boltzmann Verteilung, Entropie, Irreversibilität
	\end{itemize}
		\textbf{Wenn man eine Fettgedruckte Größe in den Übungsaufgaben sieht ist damit ein Vektor gemeint}
\part{Vorlesung 1}
	$\#$ = Anzahl
	\section{Grundlagen}
	\subsection{Vektorrechnung}
		Vektoren sind gerichtete Größen deren Komponenten bei Drehung oder allgemeiner Koordinaten Transformationen gewisse transformationseigenschaften besitzen. \\
		Skalare sind dabei invariant unter Koordinatentransformation. (z.B: Masse, Ladung, Temperatur)\\
		Ortsvektor $\vec{r}$, Geschwindigkeit $\vec{v} = \frac{d\vec{r}}{dt} = \frac{d^2\vec{r}}{dt^2} $\\
		Vektoren haben eine Richtung und eine Länge.
		\begin{center}
			\includegraphics[scale=0.5]{IMG_F483D93DB640-1.jpeg}
		\end{center}
	\subsection{Formale Schreibweise Ableitungen}
		Ableitung können sowas als $f`$ geschrieben werden als auch als $\dot{f}$ somit kann die Geschwindigkeit $\vec{v}$ als die Ableitung der Position ausgedrückt werden $\vec{v}$ = $\vec{\dot{x}}$.
	\subsection{Drehung}
	$\vec{a} = (a_x, a_y) = (x, y)$\\
	$\vec{a} = (a_u, a_v) = (u, v)$ dabei ist für u und v das Koordinatensystem einfach nur um einen bestimmten Winken $\varphi$ gedreht.\\
	\[u = x cos \varphi + y sin \varphi\]
	\[v = - x sin \varphi + y cos \varphi\]
	Länge von: \[\vec{a} = \sqrt{u^2+ v^2} = [\left(x cos \varphi + y sin \varphi)^2 (- x sin \varphi + y cos \varphi)^2\right]^\frac{1}{2}\]
	\[	[x^2 cos^2 \varphi + y^2 sin^2\varphi+xy cis \varphi sin \varphi + x^2 sin^2 \varphi + y^2 cos^2 \varphi - 2xy sin \varphi cos \varphi]^\frac{1}{2}\]
	\[ [x^2(cos^2\varphi + sin^2 \varphi) + y^2 sin^2\varphi + cos^2 \varphi]^\frac{1}{2} \]
	\[ \sqrt{x^2 + y^2} \] \qed \\
\subsection{Drehung als Matrix Multiplikation:} 
$\left( \begin{array}{c}
	u \\ 
	v
\end{array} \right)
= \left( \begin{array}{cc}
	cos \varphi & sin \varphi \\ 
	-sin \varphi & cos \varphi
\end{array} \right)
  \left( \begin{array}{ccc}
 	x \\ 
 	y
 \end{array} \right)
= \frac{2}{2} = A_{ij} x_j $  

\part{Vorlesung 2}
	\subsection{Basis der Vektorrechnung}
	Mathematisch abstrakt ist ein linearer oder Vektorraum ein "Körper" von Elementen $\vec{a}, \vec{b}, \vec{c}$, im dem eine Addition und eine ultiplikation mit skalaren $\alpha$ definiert ist.
		\subsubsection{Basis ausrechnung}
			\begin{itemize}
				\item Die Anzahl der Vektoren stimmt überein mit der Dimension des Vektorraumes.
				\item Die Vektoren sind linear unabhängig.
			\end{itemize}
		\subsubsection{Addition}
		\[	\left(\begin{array}{c}
				1\\ 
				2
			\end{array} \right) + \left(\begin{array}{c}
			3\\ 
			5
		\end{array} \right) =	\left(\begin{array}{c}
		1 + 3\\ 
		2 + 5
	\end{array} \right) = \left(\begin{array}{c}
	4\\ 
	7
\end{array} \right) \]
		\subsubsection{Subtraktion}
		\[	\left(\begin{array}{c}
				3\\ 
				4
			\end{array} \right) - \left(\begin{array}{c}
			2\\ 
			5
		\end{array} \right) =
	\left(\begin{array}{c}
		3 - 2\\ 
		4 - 5
	\end{array} \right) = \left(\begin{array}{c}
	1\\ 
	-1
\end{array} \right) \]
		\subsubsection{Multiplikation}
		\[	a * \vec{v} = 5 * \left(\begin{array}{c}
				2\\ 
				1\\
				3
			\end{array} \right) = \left(\begin{array}{c}
			5 * 2 \\ 
			5 * 1 \\
			5 * 3
			\end{array} \right) =
			\left(\begin{array}{c}
			10\\ 
			5\\
			15
			\end{array} \right) 	\]  
			Bei der Division läuft das ganze dann fast genau so ab einfach nur 5 in den als zähler der jeweiligen Koordinate schreiben.
		\subsubsection{Skalarprodukt}
			\[ \vec{a} \circ \vec{b} = |\vec{a}| \circ |\vec{b}| = \begin{pmatrix} a_1 \\ a_2 \\ a_3 \end{pmatrix} \circ \begin{pmatrix} b_1 \\ b_2 \\ b_3 \end{pmatrix} = a_1 \cdot b_1 + a_2 \cdot b_2 + a_3 \cdot b_3 \]
		\subsubsection{Eigenschaften von Rechenoperationen}
		Kommutativität $\vec{a} + \vec{b} = \vec{b} + \vec{a}$\\
		Distributivität $\lambda * (\vec{a} + \vec{b}) = (\vec{a} + \vec{b}) * \lambda$\\
		Homogenität (gemischtes Assoziativgesetz) $\alpha(\vec{a} + \vec{b} = (\alpha \vec{a}) * \vec{b} =  \vec{a} * (\alpha \vec{b})$
			
		\subsubsection{Projektion auf die Richtung \vec{b}}
			$a_b := a cos(\varphi) = \vec{a} * \vec{b} \text{ wo } \vec{b} := \frac{\vec{b}}{|\vec{b}|}$ der Einheutsvektor in $\vec{b}$ Richtung ist. (Also soll hier b der Einheitsvektor sein)\\
			Abstrake Definition eines Skalarproduktes mit obigen Eigenschaften und zusätzlichen $\vec{a}*\vec{a} > 0   \exists \vec{a} \neq 0 $ (positiv definiert)\\
		\part{Vorlesung 3}
			\subsubsection{Vektorprodukt}
				$\vec{c} = \vec{a} \times \vec{b}$ \qquad $|\vec{c}| = c = ab$  \;  $ |sin \varphi|$
		\begin{center}
			\includegraphics[scale=0.3]{vektorprodukt.png}
		\end{center}
	$\Leftarrow$ c = Fläche des von $\vec{a}$ und $\vec{b}$ aufgespannten Paralellogramms Richtung bestimmt durch \textbf{Rechtsschrauben regel}: Drehe $\vec{a}$ in Richtung $\vec{b}$
		\begin{center}
		\includegraphics[scale=0.3]{vektorprodukt2.png}
	\end{center}
	\textbf{Eigenschaften:}\\
	Antikommutativität: Das heißt, bei Vertauschung der Vektoren wechselt es das Vorzeichen\\ $\vec{a}\times\vec{b} = -\, \vec{b}\times\vec{a}$\\	
	Distributivität	$\vec{a} \times (\vec{b} + \vec{c} = \vec{a} \times \vec{b} + \vec{a} \times \vec{c})$\\
	Homogenität (gemischtes Assoziativgesetz) $\alpha(\vec{a} + \vec{b} = (\alpha \vec{a}) * \vec{b} =  \vec{a} * (\alpha \vec{b})$\\
	\textbf{Sonderfälle:}\\
	$ b(\vec{a}\times\vec{b}) \times (\vec{b}\times\vec{c})  = \vec{b} \cdot \det(\vec{a},\vec{b},\vec{c}) $\\
	$ (\vec{a}\times\vec{b}) \times (\vec{a}\times\vec{c})  = \vec{a} \cdot \det(\vec{a},\vec{b},\vec{c}) $\\
	$ (\vec{a}\times\vec{b}) \times (\vec{a}\times\vec{b})  = \vec{0}$
		\subsection{Schwarzsche Ungleichung}
	$|\vec{a} * \vec{b}| \leq |\vec{a}| |\vec{b}|$
	\subsection{Komonentendarstellung}
		Definiere 3 \"orthogonale\" das heißt orthogonale Einheitsvektoren\\
			$\vec{\hat{a}} , \vec{\hat{y}}, \vec{\hat{z}}$ \\der 
			im 3-Dimensionalen euklidischen Raum, mit (Schreibregel: Der einfachheit wegen lasse ich in den 2 Zeilen das Zeichen für Einheitsvektor weg, es sollte eigentlich dabei stehen) \\
			$\vec{x} * \vec{x} = \vec{y} * \vec{y} = \vec{z} * \vec{z} = 1   ; \vec{x} * \vec{y} = \vec{x} * \vec{z} = \vec{y} * \vec{z}  = 0$\\
			und\\
			$\vec{x} \times \vec{y} = \vec{z} , \vec{y} \times \vec{z} = \vec{x}, \vec{z} \times \vec{x} = \vec{y}$\\
			\textit{(Ab diesem Zeitpunkt ist die Schreibregel wieder aufgehoben)} andere häufige Schreibweisen:\\
				$\vec{e}_x, \vec{a}_y, \vec{a}_z ; \vec{e}_1, \vec{e}_2, \vec{e}_3$
				\begin{center}
					\includegraphics[scale=0.45]{komponentendarstellung.png}
				\end{center}
			$\vec{a} = a_x \vec{\hat{x}} + a_y \vec{\hat{y}} + a_z \vec{\hat{z}} = (a_x, a_y, a_z)$ mit $a_x = \vec{a} * \vec{\hat{x}}, a_y = \vec{a} * \vec{\hat{y}}, a_z = \vec{a} * \vec{\hat{z}}$
				\begin{center}
					I	\includegraphics[scale=0.4]{Komponentendarstellung2.png}
				\end{center}
\subsection{Spatprodukt}
\begin{center}
	\includegraphics[scale=0.3]{1000px-Parallelepiped2.png}
\end{center}
Das Spatprodukt $(\vec a, \vec b, \vec c)$ dreier Vektoren $\vec{a}, \vec{b}$ und $\vec{c}$ des dreidimensionalen euklidischen Vektorraums $\R^3$ kann wie folgt definiert werden: \\
	$(\vec{a},\vec{b},\vec{c}) = (\vec{a} \times \vec{b}) \cdot \vec{c}$
		\subsubsection{Eigenschaften des Spatprodukts}
			\begin{itemize}
				\item Das Spatprodukt ist \textbf{nicht kommutativ}. Der Wert ändert sich jedoch nicht, wenn man die Faktoren zyklisch vertauscht:\\
				$(\vec{a} \times \vec{b}) \cdot \vec{c} = (\vec{b} \times \vec{c}) \cdot \vec{a} = (\vec{c} \times \vec{a}) \cdot \vec{b}$
				\item Man kann das \textbf{Spatprodukt} \textbf{mit Hilfe der Determinante berechnen}. Für\\ $\vec a = \begin{pmatrix} a_1 \\ a_2 \\ a_3 \end{pmatrix},\ \vec b = \begin{pmatrix} b_1 \\ b_2 \\ b_3 \end{pmatrix},\ \vec c = \begin{pmatrix} c_1 \\ c_2 \\ c_3 \end{pmatrix}$ gilt:\\ $(\vec{a},\vec{b},\vec{c}) = \det\begin{pmatrix} a_1 & b_1 & c_1 \\ a_2 & b_2 & c_ 2 \\ a_3 & b_3 & c_3\end{pmatrix} =
				\det\begin{pmatrix} a_1 & a_2 & a_3 \\ b_1 & b_2 & b_ 3 \\ c_1 & c_2 & c_3 \end{pmatrix}$
				\item Die \textbf{Multiplikation} mit einem Skalar $\alpha \in \mathbb{R}$ ist \textbf{assoziativ }\\
					$( \alpha \cdot \vec{a}, \vec{b}, \vec{c} ) = \alpha \cdot ( \vec{a}, \vec{b}, \vec{c} )$
				\item Es gilt ein \textbf{Distributivgesetz:}\\
				$( \vec{a}, \vec{b}, \vec{c} + \vec{d} ) = ( \vec{a}, \vec{b}, \vec{c} ) + ( \vec{a}, \vec{b}, \vec{d} )$
				\item Invarianz unter zyklischer Vertauschung: \\
				$(\vec{a} \times \vec{b}) * \vec{c} = (\vec{b} \times \vec{c}) * \vec{a} = (\vec{c} \times \vec{a} * \vec{b})$\\
				Dies ist auch in 3-Dimensionen möglich, wie man am Beispiel der Determinante gesehen hat.
			\end{itemize}
		\subsection{Doppeltes Vektorprodukt}
			Im algemeinen nicht assoziativ\\
			Es gibt die \textbf{bac-cab-Regel}\\
			$\vec{a} \times (\vec{b} \times \vec{c})= \vec{b} (\vec{a} * \vec{c}) - \vec{c}(\vec{a} *\vec{b})$\\
			Daraus folgt auch die  Jacobi-Identität \\
			 	$\vec{a}\times(\vec{b}\times\vec{c}) +\vec{b}\times (\vec{c}\times\vec{a}) +\vec{c}\times (\vec{a}\times\vec{b}) = 0$
	\subsection{Das Differential}
		Ist $f(x)$ differenzierbar bei x, so nennt man $f^\prime(x) h$ für beliebige h  Differential  von $f(x)$. Ma schreibt oft
			\begin{center}
				$dy = df(x) = f^\prime(x)dx$
			\end{center}
			In der Mathematik wird dies Differentialform genannt, die man formal als dual zum Vektorraum bestehend aus Koordinaten $x, ...$ definiert.
	\subsection{Vektorfunktionen}
		z.B. Kurve eines Massenpunktes wird beschrieben durch $\vec{r}|t| = (x(t), y(t), z(t))$ wobei t die Zeit ist.\\
				\begin{figure}[htbp]
					\begin{minipage}[t]{10cm}
						\vspace{0pt}
						\centering
						\includegraphics[scale=0.5]{Vektorfunktionen.png}
					\end{minipage}
					\hfill
					\begin{minipage}[t]{10cm}
						\vspace{0pt}
						$\vec{\dot{r}} = \frac{d\vec{r}}{dt}\\ = \lim_{\Delta t\to\ 0} \frac{\vec{r}(t + \Delta t) - \vec{r}(t)}{\Delta t} \\= (\frac{dx}{dt}, \frac{dy}{dt}, \frac{dz}{dt}) =$ Geschwindigkeit $\vec{v}$\\
						Beschleunigung = $\vec{a} = \frac{d\vec{v}}{dt} = \frac{d^2\vec{r}}{dt^2}$\\
					\end{minipage}
				\end{figure}
				\textbf{Kettenregel, Produktregel etc. gelten auf für Vektorfunktionen.}\\
				z.B.\\
				\[\frac{d}{dt}(\vec{a} \times \vec{b}) = \frac{d\vec{a}}{dt} \times \vec{b} \vec{a} \times \frac{d\vec{b}}{dt}\]
				\[ \frac{d}{dt}\vec{r} (f(t^\prime)) = \frac{d\vec{r}}{dt}(f(t^\prime)) * \frac{df}{dt^\prime}(t^\prime) \]
	\subsection{Integration}
	\includegraphics[scale=0.05]{integral.png}
	Die Integralrechnung ist aus dem Problem der Flächen- und Volumenberechnung entstanden.
	Dabei kann die Funktion auch ein komplett random geformter klotz in einem koordinaten System sein.\\
	$\int_a^b f(x)\,\mathrm dx$\\
	\begin{figure}[htbp]
		\begin{minipage}[t]{6cm}
			\vspace{0pt}
			\centering
			\includegraphics[scale=0.45]{Integrationbeispiel.png}
		\end{minipage}
		\hfill
		\begin{minipage}[t]{6cm}
			\vspace{0pt}
			Eine der am  häufigsten genutzen Regeln in der Physik ist, dass Konstanten vor das Integral gezogen werden können.
		\end{minipage}
	\end{figure}
\subsection{Differentialrechnung}
	\begin{center}
		$dy = df(x) = f^\prime(x)dx$
	\end{center}
	In der Mathematik wird dies Differentialform genannt, die man formal als dual zum Vektorraum bestehend aus Koordinaten $x, ...$ definiert.
	\newpage
	
	\part{Vorlesung 4}
	\subsection{Vektorfunktionen}
	z.B. Kurve eines Massenpunktes wird beschrieben durch $\vec{r}|t| = (x(t), y(t), z(t))$ wobei t die Zeit ist.\\
	\begin{figure}[htbp]
		\begin{minipage}[t]{10cm}
			\vspace{0pt}
			\centering
			\includegraphics[scale=0.4]{Differentialrechnung.png}
		\end{minipage}
		\hfill
		\begin{minipage}[t]{10cm}
			\vspace{0pt}
			$f^\prime(x_0) = \frac{df}{dx} |_{x_0}$$ \\ = lim_{\Delta x \to\ 0} \frac{\Delta f}{\Delta x}\\ = lim_{x \to\ x_0} \frac{f(x) - f(x_0)}{x - x_0}$
		\end{minipage}
	\end{figure}
\subsection{Taylor Entwicklung}
	Im 1. Ordnung: $f(x) \approx f(x_0) + f^\prime(x_0)(x-x_0)$\\
\textbf{Verallgemeinerung}:\
	\[ f(x)= \sum_{n=0}^{\infty}\frac{f^(n) (x_0)}{n!} (x-x_0)^n = f(x_0) + f^\prime (x_0) (x - x_0) + \frac{f^{\prime\prime}(x_0)}{2} (x-x_0)^2 + ... \]
	\subsubsection{Potenzreihe}
	Jede \textbf{Potenzreihe} 
	$\sum_{n=0}^{\infty}a_n(x-x_0)^n$ und damit auch jede \textbf{Taylor-Reihe} hat einen Konvergenzradius $\lambda$, so daß $\sum_{n=0}^{\infty}a_n(x-x_0)^n$ für alle $|x-x_0| < r$ konvergiert. Im allgemeinen ist r durch den Abstand zur nöchsten Singularität von $f(x)$ gegeben, diese Tatsache gilt aber nur allgemein im Raum der komplexen Zahlen.\\
	\textbf{Beispiel}:\\ 
	$f(x) = \frac{1}{1 + x^2} = \sum_{n=0}^{\infty} (-1)^n x^{2^n} = \sum_{n=0}^{\infty} a_nx^n $ hat den Konvergenzradius $r = 1$, da $\frac{a_{2n+2}}{a_{2n}} = -x^2$ obwohl $\frac{1}{1+x^2}$ für alle $x \in \R$ wohldefiniert und unendlich oft differenzierbar ist. \\
	Der tiefere Grund ist daß $ \frac{1 }{1 + x^2} $ bei $x = \pm i$ singulär ist. 
	\subsection{Partielle Differentiation}
		Funktionen mehrerer Variablen, z.B. f(x,y,z), können nach den einzelnen Variablen differenziert werden, wobei man sich die anderen Variablen konstant vorstellt.\\
		\[ \frac{\partial f}{\partial x}(x,y,z) =  lim_{\Delta x \to\ 0} \frac{f(x+\Delta x,y,z - f(x,y,z))}{x,y,z} \]
		und analog für die anderen Variablen.\\ \textbf{Kurzschreibweise}:\\
		\[ f_x = \frac{\partial f}{\partial x} \quad f_y = \frac{\partial f}{\partial y} \quad f_z = \frac{\partial f}{\partial z} \] \\
		\subsubsection{Höhere Ableitungen}
		\[  f_x = \frac{\partial}{\partial x}(\frac{\partial f}{\partial x}) = \frac{\partial^2f}{\partial x^2} \\
	 f_{xy} = \frac{\partial}{\partial y} (\frac{\partial f}{\partial x}) = \frac{\partial^2 f}{\partial_y \partial_x} \]
		 übrigens gilt: $\frac{\partial^2 f}{\partial y \partial x } = \frac{\partial^2 f}{\partial_x \partial_y}$ (Symmetrie)\\
		 Man kann leicht die erweiterte Kettenregel zeigen:\\
		 Für $f(t) = f(x|t) , y(t) , z(t)$ gilt $\frac{df}{dt} = \frac{\partial f}{\partial x} \frac{dx}{dt} frac{\partial f}{\partial y} \frac{dy}{dt} + frac{\partial f}{\partial z} \frac{dz}{dt}$\\ oder als Differential \\
		 $df = \frac{\partial f}{\partial x} dx + \frac{\partial f}{\partial y} dx + \frac{\partial f}{\partial z} dz   $\\
		 Kann auch für die 1. Ordnung einer multi-dimensionalen Taylor-Entwicklung verwendet werden: \\
		 $ f(x,y,z) = f(x_0, y_0, z_0) + \frac{\partial f}{\partial x} (x_0, y_0, t_0)(x-x_0) + \frac{\partial f}{\partial y} (x_0, y_0, t_0)(y-y_0) + \frac{\partial f}{\partial z} (x_0, y_0, t_0)(z-z_0) + ... $
	
		\section{Vektoranalysis}
			Ein \textbf{Vektorfeld} ist eine Vektor-wertige Funktion mehrerer Variablen:
			\[ \vec{F}(\vec{r}) = (F_x(x,y,z), F_y(x,y,z), F_z(x,y,z) ) \]
			\textbf{Beispiele:} Geschwindkeitsfeld (e.g. Windgeschwindigkeit), Kraftfeld\\
			Ein \textbf{Skalarfeld} hat nur eine (skalare) Komponente:
				\[ f(\vec{r}) = f(x,y,z) \]\\
				\subsection{Gradient}
				
				\[ f \rightarrow \vec{\nabla} f = grad f = \vec{\nabla} f = \left( \frac{\partial f}{\partial x}, \frac{\partial f}{\partial y}, \frac{\partial f}{\partial z} \right) \text{\qquad} Skalarfeld \rightarrow Vektorfeld \] 
				Es gelten die üblichen Produkt- und Summenregeln:
					\[ \vec{\nabla}(f+g) = \vec{\nabla }f + \vec{\nabla}g; \vec{\nabla}(f+g)= g\vec{\nabla}f + f\vec{\nabla}g \]
				\subsubsection{Beispiel}
				 f sei Funktion der Abstand $|\vec{r} - \vec{r}_0| = f^\prime (| \vec{r}-\vec{r}_0|) \frac{x-x_0}{| \vec{r} - \vec{r}_0 |}$ \\
				\textit{Dabei gilt:} $|\vec{r} - \vec{r}_0| = \sqrt{(x-x_0)^2 + (y-y_0)^2 + (z-z_0)^2}$\\
				Dabei ist das ganze dann logischer weise nur vom Abstand abhängig\\
				$\Rightarrow \vec{\nabla} (\vec{r}) =   \left( \frac{\partial f}{\partial x}, \frac{\partial f}{\partial y}, \frac{\partial f}{\partial z} \right) = \frac{1}{|\vec{r} - \vec{r}_0|} ( x-x_0, y- y_0, z- z_0 )    \frac{\vec{r} - \vec{r}_0}{| \vec{r}-\vec{r}_0 |} = \vec{e}_{\vec{r} - \vec{r}_0}$ \\
				\[ \frac{\partial r}{\partial x} = \frac{1}{2} \frac{2(x-x_0)}{\sqrt{(x-x_0)^2 + (y-y_0)^2 + (z-z_0)^2}} = \frac{x-x_o}{|\vec{r - \vec{r}_0}|} \]
				Einheiten in Kegelkoordinaten:\\
				$\Rightarrow \vec{\nabla} = \frac{df}{dr}(| \vec{r}-\vec{r}_0 |)  \vec{\nabla}| \vec{r}-\vec{r}_0 | = f^\partial (| \vec{r}-\vec{r}_0 |) \frac{\vec{r} - \vec{r}_0}{| \vec{r}-\vec{r}_0 |}$
				z.B. $ f(\vec{r}) = c | \vec{r}-\vec{r}_0 |  $ Das Gravitationspotential zwischen zwei Teilchen der Masse $m_1$ und $m_2$\\
				 $ \phi(\vec{r}_1 , \vec{r}_2) = -\frac{G_N m_1 m_2}{| \vec{r}-\vec{r}_0 |}$ $\alpha = -1$ und $c = -G_N m_1 m_2$\\
				 
				 $\Rightarrow  \vec{\nabla} \phi (| \vec{r}-\vec{r}_0 |) = \frac{G_N m_1 m_2}{| \vec{r}-\vec{r}_0 |} \frac{\vec{r} - \vec{r}_0}{| \vec{r}-\vec{r}_0 |} $ 
				 Nebenrechnung: $\phi^\partial (r) = + \frac{G_N m_1 m_2}{r^2}\\
				 \vec{F}_{12} = - \vec{\nabla} \phi = - \frac{G_N m_1 m_2}{| \vec{r}-\vec{r}_0 |}(\vec{r} - \vec{r}_0)$ = Die Kraft die Teilchen 2 auf Teilchen 2 ausübt.
				
				\subsection{Rotation}
				Hierbei muss extrem auf die Vorzeichen aufgepasst werden. Es gibt zudem nur 2 Kombinationen wie man im laufe sehen wird die sich nicht wegkürzen.\\
					rot($\vec{F} (\vec{r}) = \vec{\nabla} \times \vec{F} = \sum_{i, j, k}^{}$ $\frac{\partial f_j}{\partial x_i}  \vec{e}_{i,j} = (\frac{\partial f_z}{\partial y} - \frac{\partial f_y}{\partial z}, \frac{\partial f_x}{\partial z} - \frac{\partial f_z}{\partial x}, \frac{\partial f_z}{\partial x} - \frac{\partial f_x}{\partial y}$)\\
					Es sind 3 da wir uns im 3 Dimensionalen befinden.  \quad  Vektorfeld $\Rightarrow$ Vektorfeld ; Wirbelstärke\\
					
					\subsubsection{Summe und Rotation}
						$\vec{\nabla} \times(\vec{F} + \vec{G}) = \vec{\nabla} \times \vec{F} + \vec{\nabla} \times \vec{G}$\\
						$ \vec{\nabla} \times(f \vec{F} )= f \vec{\nabla} \times \vec{F} (\vec{\nabla} f) \times \vec{F} $\\
					\subsubsection{Beispiel:}
							$f (\vec{r}) = \vec{F} = f(r) \vec{\nabla} \times \vec{r} + (\vec{\nabla} f) \times \vec{r}$\\
							$\vec{f} (\vec{r}) = (\frac{\partial f_z}{\partial y} - \frac{\partial f_y}{\partial z}, \frac{\partial f_x}{\partial z} - \frac{\partial f_z}{\partial x}, \frac{\partial f_z}{\partial x} - \frac{\partial f_x}{\partial y)} = \vec{0}  $ \qquad $\vec{r} = (x,y,z)$\\
							$\vec{\nabla}f = f^\prime \frac{\vec{r}}{|\vec{r}|} \Rightarrow (\vec{\nabla}f) \times \vec{r} = \frac{f^\prime}{|\vec{r}|} \times \vec{r} = 0$\\
							
							\subsubsection{Beispiel B}
							$\vec{F}(\vec{r}) = \vec{w} \times \vec{r}$\\
							$| \vec{F}(\vec{r})| = |\vec{w}| \sqrt{x^2+y^2}$\\
							$(\vec{\nabla}  \times \vec{F})_x = \frac{\partial f_z}{\partial y} - \frac{\partial f_y }{\partial z}  = \frac{\partial}{\partial y} (w_x y - w_y x) - \frac{\partial}{\partial z}(w_y x - w_x y)$ \\
							Damit gilt: $\vec{F} = \vec{w} \times \vec{a} \Rightarrow F_z = w_xy - w_yx$ und $F_y = w_zx - w_xz$\\
							Also ist es am ende: $\Rightarrow \vec{\nabla} \times \vec{F} = 2\vec{w}$\\
							
							
							$\Leftrightarrow \vec{\nabla} \times (\vec{w} \times \vec{r})= 2\vec{w}$
							
					\subsection{Divergenz}
						div $\vec{f} = \vec{\nabla}, \vec{F} = (\frac{\partial}{\partial x}, \frac{\partial }{\partial y} , \frac{\partial }{\partial z}) * F_x , F_y, F_z = \frac{\partial F_x}{\partial x} + \frac{\partial F_y}{\partial y}+ \frac{\partial F_z}{\partial z}$\\
						Vektorfunktion $\rightarrow$ Sklarafunktion\\
						$\vec{\nabla } (\vec{F} + \vec{G}) = \vec{\nabla} \vec{F} + \vec{\nabla} \vec{G} $ \quad $ \vec{\nabla}+ (f \vec{f}) = f \vec{\nabla} * \vec{F} +(\vec{\nabla f}) * \vec{F} $\\
						\subsubsection{Beispiele}
						$\vec{F} (\vec{r}) = \vec{r}$ \quad $\vec{\nabla} * \vec{r} = 3 $\\
						anderes Beispiel: $\vec{F} (\vec{r}) = \vec{w} \times \vec{r}$ \quad $\vec{w} = const.$\\
						wähle deine Einschränkungen: $\vec{w} = w\vec{e}_z = (0,0,w) \rightarrow \vec{w} \times \vec{r} = (-wy, wx, 0)$\\
						$\vec{\nabla} * (\vec{w} \times \vec{ r}) = \frac{\partial}{\partial x} (wx) + \frac{\partial}{\partial z} 0 = 0$\\
						Weitere Beispiele im im Skript.\\		
						\part{Vorlesung 5}				
					\subsection{Gradient und totales Differential}
						$f(x,y,z) \Rightarrow df = \frac{\partial f}{\partial x} dx +\frac{\partial f}{\partial y} dy+ \frac{\partial f}{\partial z} dz = (\vec{\nabla} f) * d\vec{ r}$\\
						$\vec{\nabla} f = \frac{\partial f}{\partial x}, \frac{\partial f}{\partial y} , \frac{\partial f}{\partial z}  \quad d\vec{ r} = (dx,dy,dz)$\\
						$\frac{df}{dt} = (\vec{\nabla} f) * \frac{d\vec{ r}}{dt} $\\
						$\Rightarrow$ $\int_{\vec{ r}_0}^{\vec{ r}_1} \frac{df}{dt} dt = \int_{\vec{ r}_0}^{\vec{ r}_1} (\vec{\nabla } f) d\vec{ r} = f(\vec{ r}_1) - f(\vec{ r}_0)$\\
						\subsection{Wegintegral eines Vektorfeldes $\vec{ F}(\vec{ r})$}
							Angenoimmen wir haben eine Kurve und wollen ein Kurvenintergral bilden und nennen dies dann c. Dabei ist der eine Endpun kt $\vec{ r}_0$ und $\vec{ r}_1$\\
							$\int_{c}^{} \vec{ F}(\vec{ r}) *d\vec{ r} = \int_{c}^{} [ F_x dx + F_y dy + F_z dz ] :=  \\ \lim_{N \to\ \infty}
							\sum_{i =1}^{N} \vec{ F}(\vec{ r}_i) * \Delta\vec{ r}_i = \lim_{N \to\ \infty} \sum_{i =1}^{N}\vec{ F}(\vec{ r}_i) * \frac{ \Delta \vec{ r}_i}{\Delta t} \Delta t$\\
							$\lim_{\Delta t\to\ 0} \Delta t_i = \lim_{N \to\ \infty} \Delta \vec{ r}_i = 0$\\
							Also zum Beispiel eine Arbeit, die durch einer Kraft $\vec{ F}(\vec{ r}) * d\vec{ r}$ verrichtet wird
								\[ W = \int \vec{ F}(\vec{ r}) * d\vec{ r} \]
								Dabei muss man sich immer den Kontext klar machen da sich dadurch ganz einfach Vorzeichen ändern können also der unterschied ob Arbeit verrichtet werden muss oder nicht. \textit{Zum Beispiel das Skalarprodukt würde sich daruch komplett ändern}\\
								\[ \int_{c_1} \vec{ F}(\vec{ r}) * d\vec{ r} \]
								\[ \int_{c_2} \vec{ F}(\vec{ r}) * d\vec{ r} \]
								diese Intergrale sind wegunabhängig $\Leftrightarrow \vec{ F} = \vec{\nabla } f \Leftrightarrow \vec{\nabla } \times \vec{ F} = 0$ \\
								\subsubsection{Eigenschaften:}
								$ \int_{-c} \vec{ F}* d\vec{ r} = -\int_{c} \vec{ F}*d\vec{ r}  $\\
								$ c = c_1 + c_2 \Rightarrow \int_{c}\vec{ F}*d\vec{ r} + \int_{c_2}\vec{ F}* d\vec{ r} $ 
					\section{Grundlagen der Dynamik}
						\textbf{Axiome der Newtonschen Mechanik}\\
						\begin{itemize}
							\item \textbf{1. Trägheitsgesetz:} Körper, auf den keine Kräte wirken, bewegt sich geleichförmig und gleichlinig, das heißt $ \vec{v} = const , \vec{v} = \vec{0}$ ist ein Spezialfall.
							\item \textbf{2. Aktionsprinzip} Die Zeitliche Änderung des Impulses $\vec{p}$ eines Körpers ist gleich der auf ihn einwirkenden Gesammtkraft
							\[ Impuls = m*\vec{v} = m\frac{d\vec{ r}}{dt} = \vec{p} \]
							\[ \vec{ F}_{total} = \frac{d\vec{p}}{dt} = \dot{\vec{p}} = m \dot{\vec{v}} = m \ddot{\vec{ r}} = m \vec{a} \]
							\item \textbf{3. Actio= Reactio} die von Körper 1 auf Körper 2 ausgeübte Kraft $\vec{ F}_{21}$ ist gleich dem negativen der von Körper 2 auf Körper 1 ausgeübte Kraft 
							\[ \vec{ F}_{21} = - \vec{ F}_{12} \]
							\item \textbf{Superpositionsprinzip} Kräfte addieren sich vektoriel.
						\end{itemize}
							In Newton$^\prime$s Mechanik sind \textbf{Raum} und \textbf{Zeit} absolute Begriffe und die Zeit läuft immer gleich ab unabhängig vom Inertialsystem,  was natürlich in konflikt mit der Relativitätstheorie steht.\\
							Newtons Axiome gelten zunächst nur in unbeschleunigten, sog. Inertialsystemen. Intertialsysteme bewegen sich relativ zueinander mit konstanter Geschwindigkeit\\
							\subsection{Allgemeine Galilei- Transformation zwischen Inertialsystemen}
								\[ \vec{ r}^\prime = \vec{ r} + \vec{v}_0 * t + \vec{ r}_0 \qquad \vec{v}_0 = const ; \vec{ r}_0 = const \]
								Wie man sieht transformieren sich damit die Ortskoordinaten aber die Zeit bleibt offensichtlicherweise gleich.$t^\prime = t $ \
								\[ \vec{v}^\prime = \frac{d\vec{ r}^\prime}{dt} = \frac{d\vec{ r}}{dr} + \vec{v}_0 = \vec{v} + \vec{ v}_0 \]\
								\[ \vec{ a}^\prime = \frac{d^2\vec{ r}^\prime}{dt^2} = \frac{d^2\vec{ r}}{dt^2} = \vec{ a} \]
							\subsubsection{Galileisches Relativitätsprinzip}
								Newtonsche Bewegungsgleichung ist forinvariant unter Galileitransformation.
								\subsubsection{Exkurs Kosmologie}
									In der Kosmologie gibt es als \textit{bevorzugtes} Bezugssystem das Rugesystem der thermischen Mikrowellenhintergrundstrahlung.
									\subsubsection{Exkurs Machsches Prinzip}
										Existenz von Raum, Zeit und Intertialsystem ist beinflusst durch die Massenverteilung auf sehr großen (kosmologischen) Skalen.
									\subsubsection{Ausblick}
										\textbf{spezielle Relativitätstheorie:} Raum und Zeit werden relativ und die Galileitransofrmation werden durch Lorentztransformation ersetzt.\\
										\textbf{allgemeine Relativität}  Raum und Zeit (Geometrie) sind an die Materieverteilung gekoppelt. Das hat dann die Folge, dass die Teilchenbewegung bestimmt wird durchd ie Geometrie von Raum und Zeit. Die Massenverteilung bestimmt dann aber erst die Geometrie von Raum und Zeit.
						\part{Vorlesung 6}
							\subsection{Koordinatentransformationen - Krumlinige (räunliche) Koordinaten}
								Man habe n-Dimensionen $ (x_i) (y_i) \qquad |\leq i,y \leq n  $	
								\[ x_i = x_i(y_{y_1, ..., y_n}) = x_i(y_i) \]
								\[ dx_i = \sum_{j = 1}^{n} \frac{\partial x_i}{\partial y_y} dy_i \]
								\subsubsection{Beispiel}
									$ n=3$ mit $x_i =(x,y,z)$ Euklidische Kooridnaten; und $y_i = (u,v,w)$\\
									$x= x(u,v,w) \qquad dx = \frac{\partial x}{\partial u}du + \frac{\partial x}{\partial v} dv + \frac{\partial x}{\partial w} dw$\\
									$y = y(u,v,w) \qquad dy = \frac{\partial y}{\partial u}du + \frac{\partial y}{\partial v} dv + \frac{\partial y}{\partial w} dw$\\
									$y = z(u,v,w) \qquad dz = \frac{\partial z}{\partial u}du + \frac{\partial z}{\partial v} dv + \frac{\partial z}{\partial w} dw$\\
									$\vec{r} = (x_1, ... , x_n) = (x_i) d\vec{ r}= (dx, ... , dx_n) = (dx_i) = \sum_{i} dx_i \vec{e}_i$ mit $ \vec{e}_i, \vec{e}_j = \delta_{i,j}$\\
									In allgemein krummlinigen Kooridianten $(y_i)$ wird die damit
									\[ d\vec{r} = \sum_{i}\frac{\partial \vec{ r}}{\partial y_i} dy_i \]
									z.B. in 3 Dimensional\\
									\[ d\vec{ r} =  (dx,dy,dz) = \vec{e}_1 dx + \vec{e}_2dy + \vec{e}_3dz \]
									\[ = \frac{\partial \vec{r}}{\partial u}du + \frac{\partial \vec{r}}{\partial v} dv + \frac{\partial \vec{r}}{\partial w}dw \text{ mit } \frac{\partial \vec{r}}{\partial u} = \left( \frac{\partial x}{\partial u},  \frac{\partial y}{\partial u},  \frac{\partial z}{\partial u} \right) etc.  \]
									Man kann neue Basis-Einheitsvektoren
										\[ \vec{u}_{12} := \frac{1}{b_{12}} \frac{\partial \vec{r}}{\partial y_{12}} \quad \text{ mit } b_{k}  := \left| \frac{\partial \vec{r}}{\partial y_{12}} \right| \]
										(k = 1,2,3) \qquad definieren, so daß
										\[ d\vec{r} = \sum_{n} \vec{u}_nds_k \text{ mit den Längenelementen } ds_k = k_k dy_k \]
										$= \sum_{k} \frac{\partial \vec{r}}{\partial y_{12}} dy_{12}$
										Man nennt $(y_i)$ ein orthogonales Koordiantensystem wenn:
										\[ \vec{U}_i \times \vec{U}_j = 0 \text{ für } i \neq j \text{und damit auch } \vec{U}_i * \vec{U}_j = \delta_{ij} \text{ an jedem Raumpunkt liegt. } \]
										Dann ist das Flächenelement, da von zwei Seiten der Länge $ds_{i}, ds_j $ angespannt wird, gegeben durch: 
										\[ dF_{ij} = ds_i, ds_j  = b_i b_j dy_i dj_j \]
										Man kann sich das so vorstellen wie ein Rechteck was grade aufgespannt wird.\\
										Und das Volumenelement
										\[ dV = \prod_{i=1}^{n} ds_i = \prod_{i'=1}^{n} b_i dy_i \]
							\subsection{Ebene: Polarkoordianten}
								Wir befinden uns als Beispiel zur besseren veranschaulichung nun im $\R^2$, also im 2 Dimensionalen Raum. $n=2$\\
								$ x = r cos \varphi \qquad y = r sin \varphi $
							\begin{figure}[htbp]
								\begin{minipage}[t]{10cm}
									\vspace{0pt}
									\centering
									\includegraphics[scale=0.35]{screenshot001.png}
								\end{minipage}
								\hfill
								\begin{minipage}[t]{10cm}
									\vspace{0pt}
								$ \vec{r} = (r cos \varphi, r sin \varphi)$
								\end{minipage}
							\end{figure}\\
							\subsubsection{Umdrehung}
							 $ r = \sqrt{x^2 + y^2}, \varphi = \pm arctan \frac{y}{x} \text{ für } y  \neq 0$
							 für $\varphi \in [ 0 2 \pi[ ,[ -\pi + \pi[ $ 
							\[ \frac{\partial \vec{r}}{\partial r} = (cos \varphi, sin \varphi) \quad |\frac{\partial \vec{r}}{\partial r}| = 1 \Rightarrow \vec{U}r = \vec{e}_r= (cos \varphi, sin \varphi) \]
							\[ \frac{\partial \vec{r}}{\partial \varphi} = (-r sin \varphi, r cos \varphi) \quad |\frac{ \partial \vec{r}}{\partial \varphi}| = r \Rightarrow\vec{u}_\varphi = \vec{e}_\varphi = (- sin \varphi, cos \varphi) \]
							\subsubsection{Bahnkurve}
							$\vec{ r}(t) = r(t) \hat{\vec{e}} \Rightarrow \vec{v}(t) = \dot{\vec{r}}(t) = \dot{r}(t) \vec{e}(t) + r(t) \dot{\vec{e}_r(t)}$\\
							$ \dot{\vec{e}}_r = \frac{d \vec{e}_r}{d \varphi} \frac{d \varphi }{dt} = \vec{e}_\varphi \frac{dy}{dt} = (-sin \varphi , cos \varphi) \dot{\varphi} = \dot{\varphi} \vec{e}_\varphi $\\
							$\Rightarrow$ Wenn $\vec{ v} = v_r \vec{e}_r + v_\varphi \vec{e}_\varphi $ dann ist $v_r = \dot{r} ; v_\varphi = r \dot{\varphi}$ dabei ist $\varphi$ die Winkelgeschwindigkeit.\\
							$ \Rightarrow \dot{v} = \dot{r}(t) \vec{e}_r + r(t) \dot{\varphi}\vec{e}_\varphi \Rightarrow v_r = \dot{r} ; v_\varphi = r \dot{\varphi} $\\
							$\dot{\varphi}$ ist die Winkelgeschwindigkeit und 2 Punkt ist die Rotationsgeschwindigkeit = radius. Winkelgeschwindigkeit
							\subsubsection{Beschleunigung}	
								$\vec{ a} = \ddot{\vec{ r}} = \dot{\vec{ v}} = \frac{d}{dt} (v_r \vec{e}_r + v_\varphi \vec{e}_\varphi) = \frac{d}{dt} (\dot{r} \vec{e}_r + r\dot{\varphi} \vec{e}_\varphi) $
								\[ \vec{e}_r = \dot{\varphi} \vec{e}_\varphi \text{ und } \dot{\vec{e}}_\varphi = - \dot{\varphi}\vec{e}_r \]
								$ \ddot{r} \vec{e}_r + \dot{r} \dot{\vec{e}}_r + \dot{r} \dot{\varphi} \vec{e}_\varphi + r \ddot{\varphi} \vec{e}_\varphi $
								Nun nutzen wir folgende zuweisungen: 
								\[ \dot{\vec{e}} = \dot{\varphi} \vec{e}_\varphi \text{ und } \dot{\vec{e}}_\varphi  = \dot{\varphi} (- cos \varphi, - sin \varphi) = - \dot{\varphi} \vec{e}_r \]
								Damit ergibt sich aus der oberen Gleichung: $ (\ddot{r} - r \dot{\varphi}^2)\vec{e}_r + (r\ddot{\varphi} + 2\dot{r} \dot{\varphi}) \vec{e}_\varphi $\\
							mit $\vec{ a} = a_r \vec{e}_r + a_\varphi \vec{e}_\varphi$ also
								\[ a_r = \ddot{r} - r \dot{\varphi}^2 \text{ \quad \textbf{Radialbeschleunigung}} \]
								\[ a_\varphi = r \ddot{\varphi} + 2\dot{r} \dot{\varphi} \text{\quad \textbf{ Winkelbeschleunigung}}\] 
							\subsection{Linien- und Flächenelemente:}
								
								\begin{figure}[htbp]
									\begin{minipage}[t]{10cm}
										\vspace{0pt}
										\centering
										\includegraphics[scale=0.5]{screenshot002.png}
									\end{minipage}
									\hfill
									\begin{minipage}[t]{10cm}
										\vspace{0pt}
										$ ds_r = dr \qquad ds_\varphi = r d_\varphi $\\
										$ \Rightarrow dF = dxdy = ds_r ds_\varphi $\\
										$ = rdrd_\varphi $
									\end{minipage}
								\end{figure}
							\part{Vorlesung 7}
							\subsection{Zylinderkoordianten}
								Für Zylinderkoordinaten befinden wir uns im 3 Dimensionalen wie der name bereits vermuten lässt, $n=3$
								\[ x = \rho cos \varphi \quad y = \rho sin \varphi \quad z = z \]
								$ \rightarrow$ Ebene Polarkoordianten $+$ dazu gedrehte z-Achse
							\begin{figure}[htbp]
								\begin{minipage}[t]{10cm}
									\vspace{0pt}
									\centering
									\includegraphics[scale=0.5]{screenshot3.png}
								\end{minipage}
								\hfill
								\begin{minipage}[t]{10cm}
									\vspace{0pt}
									Umdrehung:\\
									$ \rho = \sqrt{x^2 + y^2} ; \varphi = \pm arctan \frac{y}{x}  \quad y \neq 0  $\\
									$ z = z $
								\end{minipage}
							\end{figure}
						Man findet die orthogonalen Einheitsvektoren
						\[ \vec{e}_\rho = (cos \varphi, sin \varphi, 0) \quad \vec{e}_\varphi= (-sin \varphi, cos \varphi, 0) \quad \vec{e}_z = (0,0,1) \]
						Ferner $dV = ds_\rho ds \varphi ds_z = \rho d\rho d\varphi dz$\\
						\subsubsection{Geschwindkeit und Beschleunigung}
							Für Geschwindkiet und Beschleunigung erhielt man
							\[ \vec{ v} = \dot{\rho} \vec{e}_\rho + \rho\dot{\varphi} \vec{e}_\varphi + \dot{z} \vec{e}_z \]
							\[ \vec{ a} = (\ddot{\rho} - \rho \dot{\varphi}^2) \vec{e}_\rho + (2 \dot{\rho} \dot{\varphi} + \rho \ddot{\varphi}) \vec{e}_\varphi + \ddot{z} \vec{e}_z \]
					\subsection{Kugelkoordinaten - sphärische Polarkoordinaten}
						Wir bleiben im 3 Dimensionalen, n=3 \\
						\[ x = r sin \Theta cos \varphi \]
						\[ y = r sin \theta sin \varphi \]
						\[ z = r cos \theta \]
					\subsubsection{rotierendes Bezugssystem und Scheinkräfte}
						\begin{center}
							\includegraphics[scale=0.5]{screenshot004.png}
						\end{center}
							Für jeden Vektor $\vec{b}$ gilt: $\vec{b} = \sum_{i} bi \vec{e}_i = \sum_{i} b_i^\prime \vec{e}_j^\prime$\\
							$\Rightarrow \frac{d \vec{b}}{dt} = \sum_{i}\frac{db_i}{dt} \underbrace{\vec{e}_i}_{\text{Zeitunabhngig}} = \sum_{i} \frac{dbi^\prime}{dt} \vec{e}_i^\prime = \sum_{b_i}^\prime \underbrace{\frac{d\vec{e}_i^\prime}{dt}}_{ \vec{w} \times \vec{e}_i^\prime} = \underbrace{\left( \frac{d \vec{b}}{dt} \right)^\prime}_{Ableitung im \sum^\prime-System} + \vec{w} \times \vec{b} $ \\
						\textbf{Ferner:}\\
							$\vec{r}(t) = \vec{r}_0(t) + \vec{r}^\prime(t)  \Rightarrow \vec{v}=\frac{d\vec{r}}{dt} = \dot{\vec{r}_0}+\left( \frac{d \vec{r}^\prime}{dt} \right)^\prime + \vec{w} \times \vec{r} \\\dot{\vec{r}}_0 + \underbrace{\vec{ v}^\prime}_{Geschwindigkeit im \sum^\prime-System} $\\
							Notwendige Differentiation ergibt Tranforamtionsgesetzt für die beschleunigung:\\
							$ \vec{a} = \frac{d^2\vec{ r}}{dt^2}= \frac{d\vec{ v}}{dt}= \ddot{\vec{ r}}_0 + 
							(\frac{d \vec{ v}^\prime}{dt})^\prime + \vec{w} \times \vec{v} + \dot{\vec{w}} \times \vec{ r}^\prime + \vec{w}((\frac{d\vec{ r}^\prime}{dt})^\prime + \vec{w} \times \vec{ r}^\prime) $\\
							$ \ddot{\vec{ r}}_0 + \vec{a}^\prime + 2\vec{w} \times \vec{ v}^\prime + \vec{w} \times (\vec{w} \times \vec{ r}^\times) +\dot{\vec{w}} \times \vec{ r}^\prime$\\ 
							wobei $\vec{a}^\prime = (\frac{d\vec{v}^\prime}{dt})^\prime$ die Beschleunigung gemessen im $\sum^\prime-System$ ist Eingesetzt in $\vec{F} = m \vec{a}$, damit bekommt man letztendlich dann:
							\[ m\underbrace{\left( \frac{d^2 \vec{r}^\prime}{dt^2} \right)^\prime }_{\vec{a}^\prime} = \vec{ F} - m [\ddot{\vec{r}}+\underbrace{2\vec{w}\times \vec{v}^\prime}_{\text{Coriolis-Kraft}} + \underbrace{\vec{w} \times (\vec{w} \times \vec{r}^{\prime\prime})}_{\text{ Zentrifugalkraft }}+\dot{\vec{w}}\times \vec{ r}^\times ]  \]		
							\subsubsection{Beispiel}
							rotierendes terrestrisch Bezugssstem: Auf einem Luftraum der sich auf die Nordhalbkugel um Nord nach Süd bewegt wirkt eine Corioliskraft die nach Westen zeigt. Deshalb drehen sich Luftmassen auf der Nordhalbkugel im Gegen-Uhrzeigersinn um Tiefdruckgebiete. Auf der Süd halbkugel ist es genau andersherum.
							\begin{center}
								\includegraphics[scale=0.5]{Weltkugel.png}
							\end{center}
						\part{Vorlesung 8}
						\subsection{Spezielle Relativitätstheorie - Lorentstransformation}
							Galilei$-$Transformationen in der Newton$^\prime$schen Mechanik fürhren zur Vektoraddition von Geschwindigkeiten: 
							\[ \vec{ r}^\prime = \vec{ r} + \vec{ v}_0 * t + \vec{ r}_0 \Rightarrow \vec{ v}^\prime = 
							\frac{d\vec{ r}^\prime}{dt} = \frac{d\vec{ r}}{dt} + \vec{ v}_0 = \vec{ v} + \vec{ v}_0 \]
							insbesondere hängt Lichtgeschwindigkeit vom Bezugssystem (Inertialsystem) ab. $\exists$ bevorzugtes Inertialsystem in welchem Lichtgeschwindigkeit $= c_0 \rightarrow$ Äther! \qquad Für den Schall ist der Äther (Medium) die Luft.
							\subsubsection{Michelsan-Versuch}
								\begin{center}
									\includegraphics[scale=0.5]{michelsan.png}
								\end{center}
							Lichtlaufzeit in Arm 1 = $\frac{l_\prime}{c_0 - v} + \underbrace{\frac{l_\prime}{c + v} }_{\text{kürzt sich}} + \frac{l_\prime}{c_0 + v} = t_1 = \frac{2 l_\prime}{c_0}\frac{1}{1-\frac{v^2}{c_0^2}}$\\
							Lichtlaufzeit in Arm 2 = $ (v\frac{t_2}{2})^2 + l_2^2 = (c_0 \frac{t_2}{2})^2 \Rightarrow t_2 =' \frac{2 l_1}{c_0} \frac{1}{\sqrt{1 - \frac{v^2}{c_0^2}}} $\\
							$ \Rightarrow \Delta t = \frac{2}{c_0}\left( \frac{l_1}{1- \frac{v^2}{c_0^2}} - \frac{l_2}{\sqrt{1- \frac{v^2}{c_0^2}}} \right) $\\
							Drehung um $ 90^\circ $ entspricht Austausch von $l_1$, und $l_2$:\\
								$\Delta t^\prime = \frac{2}{c_0} \left( \frac{l_1}{\sqrt{1- \frac{v^2}{c_0^2}}} - 
								\frac{l_2}{1- \frac{v^2}{c_0^2}} \right)$\\
								Die Differenz der Lichtlaufzeiten ändert sich damit um
								\[ \delta t = \Delta t - \Delta t^\prime = \frac{2(l_1 + l_2)}{c_0} \left( \frac{1}{1- \frac{v^2}{c_0^2}}- \frac{1}{\sqrt{1- \frac{v^2}{c_0^2}}} \right) \]
						war die Interfrequenzmuster um die Phase
						\[ \Delta \Phi = \underbrace{w}_{\text{Frequenz des \\laserlichtes}} \delta t \]
						ändert.\\
						Experimentell ist $\Delta \delta = \Delta \Phi = 0$\\
						$\Rightarrow$ \textbf{Die Vakuumlichtgeschwindigkeit ist universell und in jedem Inertialsyxstem identisch.}\\
						Damit sind alle Inertialsysteme gleichberechtigt und nur Relativgeschwindigkeiten sind phyikalisch relevant. Dies ist nicht der Fall in der Äthertheorie, wie folgendes Beispiel zeigt (Geschwindigkeiten relativ zum Äther)
						\begin{center}
							\includegraphics[scale=0.5]{Inertialsystem1.png}
						\end{center}
					Wenn $ v=0 $ und $d =Distanz$ der Beobachter-Quelle dann ist $t_0 = \frac{d}{c_0};$ wenn $v>0$ dann $d = c_0 t + v t$\\
					$\Rightarrow t= \frac{t_0}{1 + \frac{v}{c_0}} \Rightarrow$ frequenz $f = \frac{1}{t} = f_0(1+ \frac{v}{c_0}) $
					\begin{center}
						\includegraphics[scale=0.5]{Beobachterquelle.png}
					\end{center}
					$t = t_0 - \frac{v t_0}{c_0} + \frac{d}{c_0} \Rightarrow f= \frac{1}{\Delta t} = \frac{f_0}{1- \frac{v}{c_0}} $
					$\Rightarrow$ die beiden Dopplereffekte wären in zweiter Ordnung in $\frac{v}{c_0}$ verschieden.\\
					Eine Möglichkeit, die Lichtgeschwindigkeit unabhängig vom Inertialsystem zuhalten ist
					\[ c_0^2\Delta t^2 - (\Delta \vec{ r})^2 = \text{\textit{ unabhängig vom Inertialsystem}} \]
					zu setzen, denn dann ist für Bewegung mit Lichtgeschwindigkeit $ c_0^2\Delta t^2 - (\Delta \vec{ r})^2 = 0 $ in allen Inertialsystem. \\
					Gesucht ist aber eine Transformation (ohne Einschränkung mit nur einer Raumkoordinate):
					\begin{figure}[htbp]
						\begin{minipage}[t]{10cm}
							\vspace{0pt}
							\centering
							\includegraphics[scale=0.5]{Lichtgeschwindigkeit2.png}
						\end{minipage}
						\hfill
						\begin{minipage}[t]{10cm}
							\vspace{0pt}
							$x^\prime = ax + bt$\\
							$t^\prime = cx + dt $\\
							$ a,d  $
						\end{minipage}
					\end{figure}
					so daß:
					\[ (c_0 t^\prime)^2 - x^{\prime2} = (c_0 t)^2 -x^2 \]
					Definiere $ a:= \gamma$. Per Definition bewegt sich ein Punkt in Ruge in $\sum^\prime, \Delta x^\prime = 0 $, mit Geschwindigkeit v in $\sum , v = \frac{\Delta x}{\Delta t}$
					\[ 0 = \Delta x^\prime = \gamma \Delta x + b \Delta t \Rightarrow b = - \gamma \frac{\Delta x}{\Delta t} = - \gamma v \]  aber \[ x^\prime = \gamma(x - vt) \]
					weiter würde es gehen:\\
					$ (c_0t^\prime)^2 - x^{\prime2} = c_0^2(cx + dt)^2 - \gamma^2(x -vt)^2 \\
						= c_0^2(d^2 - \gamma^2 \frac{v^2}{c_0^2}) t^2 + (c_0^2 c^2 - t^2) x^2 + 2(c_0^2 cd + \gamma^2 v)x t $\\
						$\Rightarrow$
						\[ (1) d^2 = 1 + \gamma^2 \frac{v^2}{c_0^2} \]
						\[ (2) c_0^2 c^2 = \gamma^2 - 1 \]
						\[ (3) c_0^2 cd = -\gamma v \]
						multipliziere (1) und (2) \qquad $\underbrace{\Rightarrow}_{quadiere (3)} c_0^2 c^2 d^2 = \gamma^4 \frac{v^2}{c_0^2}$
						$\Rightarrow c_0^2 c^2 d^2 = - 1 + \gamma^2 (1 - \frac{v^2}{c_0^2}) + \gamma^4 \frac{v^2}{c_0^2} $\\
						Nun setzen wir gleich $\Rightarrow \gamma^2 = \frac{1}{1-\frac{v^2}{c_0^2}} \Rightarrow \gamma = \frac{1}{\sqrt{1- \frac{v^2}{c_0^2}}}$ weil $ a = \gamma > 0$\\
						$\Rightarrow d^2 = \gamma^2(\frac{1}{\gamma^2} + \frac{v^2}{v_0^2}) = \gamma^2 \qquad d = \gamma$ weil $d > 0$\\
						\quad $(3) \Rightarrow c = - \frac{\gamma^2 v}{c_0^2 d} = - \gamma \frac{v}{c_0^2}$\\
						\[\Rightarrow x^\prime = \gamma(x - vt)\quad ; \quad  t^\prime = \gamma(t - \frac{vx}{c_0^2}) \]
						Dabei entspricht $\gamma = \frac{1}{\sqrt{1 - \frac{v^2}{c_0^2}}}$
						\textbf{Hinweis:} \\Oft werden natürliche Einheiten verwendet für die $c_0 = 1$\\
						Die Koordinaten $\perp \vec{v}$ bleiben unverändert, das heißt 
						\[ y^\prime = y \qquad z^\prime = z \]
						Dabei ist zu beachten: Für $\frac{v}{c_0} \rightarrow 0 $ ist $ \gamma = 1 + 0(v^2) \Rightarrow $ In erster Ordnung in v erhält man Galilei-Transformation: 
						\[ v^\prime = x -vt + 0(v^2) \quad t^\prime = t - \frac{vx}{c_0^2} + 0(v^2) \]
						\newpage
					\part{Vorlesung 9}
						\subsection{Anwendung- Relativitätstheorie}
							 \subsubsection{Geschwindigkeitsaddition oder subtraktion}
							In System $\sum$ bewegt sich ein Teilchen mit Geschwindigkeit\\
							$ u = \frac{\Delta x}{\Delta t} \Rightarrow $ In $\sum^\prime$ ist Geschwindigkeit\\
							$u^\prime = \frac{\Delta x^\prime}{\Delta t ^\prime} = \frac{\Delta x - v\Delta t}{\Delta t - v 
								\frac{\Delta x}{c_0^2}} =($dividiere oben und unten durch $\Delta t$ und verwende $u = \Delta x / \Delta t  \frac{u - v}{1 - \frac{v u}{c_0^2}}) $\\
							Für $u,v << c_0$ gilt Galilei-Transformation bis auf Terme zweiter Ordnung:
							\[ u^\prime = u - v + 0 (v^2, u^2) \]
						\subsubsection{Zeitdilatation}
						Betrachte einen Prozess, der im $\sum$ im Ruge stattfindet und eine Zeit $\Delta t$ dauert, z.B. radioaktiver Zerfall.\\
						$ \Delta t ^\prime = \gamma \Delta t \ge \Delta t $\\
						\textbf{Vierervektor-Formatlismus:} \\
							Verwende natürliche Einheiten $c_0 \equiv 1$ der Einfachheitshalber\\
							$ \Rightarrow t^\prime = \gamma(t -vx) \quad y^\prime = y\\
							\Rightarrow x^\prime = \gamma(x - vt) \quad z^\prime = z $\\
							$(t, \vec{x})$ vildet einen Vierervektor dessen Norm $t^2 - \vec{x}^2$ erhalten ist. Analog bildet $(E , \vec{p})$ einen Energie Impuls Vierervektor.
							$ \Rightarrow E^\prime = \gamma(E - vp_x) \quad px^\prime = \gamma(p_x - vE) $\\
							$ p_y^\prime = p_y \qquad p_z^\prime = p_z $\\
							Spezialfall: Teilchen in Ruge im $\sum$, aber $\vec{p} = 0, E = m_0 = \text{\textit{ Ruhemasse}}$\\
							 $\Rightarrow E^\prime = \gamma E = \gamma m_0 \\
							 p_x^\prime = - \gamma v E = - \gamma v m_0\\
							 p_y^\prime = p_z^\prime = 0$\\
							 $\underbrace{\Rightarrow}_{\text{ ersetze durch ungestrichene Größe }} \vec{p}^2 + m_0^2 = m_0^2(1 + \gamma^2 v^2) = \frac{m_0^2}{1 - v^2}(1-v^2 + v^2) = \frac{m_0^2}{1-v^2} = \gamma^2 m_0^2 = E^2$\\
							 in alltäglichen Koordinaten: $E^2 = c_0^2 \vec{p}^2 + m_2 c_0^4 \qquad [E] = [mv^2], [p]= [mv]$\\
							Dabei zu beachten ist dass, E positiv und eine negative Wurzel hat.\\
							$E < 0$ entspricht Antiteilchen\\
							$\rightarrow$ in der Relativitätstheorie hat jedes Teilchen ein Anti-Teilchen\\
							\textbf{Nicht-relativistisches Limit:}\\
							$ E = \sqrt{m_0^2 + \vec{p}^2} = m_0 \sqrt{1 + \frac{\vec{p}^2}{m_0^2}} = m_0 (1 + \frac{1}{2} \frac{\vec{p}^2}{m_0^2} + 0 (p^4))\\
							E = \gamma m_0 = \frac{m_0}{\sqrt{1 - v^2}} - m_0 (1 + \frac{1}{2} v^2 + 0(v^2)) $\\
							$ \approx m_0 + \underbrace{\frac{1}{2} m_0 v_0^2}_{\text{ nicht relativistische kinetische Energie}}$
						\part{Vorlesung 10}
							\subsection{Allgemeine Form der Newtonschen Bewegungsgleichung:}
								m$\ddot{\vec{ r}} = \vec{f}(\vec{r} , t) \rightarrow 
								\text{ gewöhnlicghe Differentialgleichung zweiter. Ordnung}$\\
								$m \ddot{x}_i = F_i(x_1,x_2,x_3,t) \qquad i= 1,2,3$\\
								\subsection{Mehrteilchensysteme für N Teilchen}
									$m_j  \ddot{\vec{ r}}_j = \vec{ F}_j (\vec{ r}_{11} ... , \vec{ r}_{N,t}) \quad j = 1, ... , N$\\
									$\Rightarrow 3N$ Differentialgleichung 2ter Ordnung $\rightarrow 3N$ Freiheitsgrade.\\
									\textbf{Siehe Dokument im Physik 1 Ordner der Cloud: Differentialgleichung.pdf} 
						\subsection{Phasenraum}
							Für n Freiheitsgrade hat man den n-dimensionalen Ortvektor $\vec{r} = (y_1, ..., y_n)$; Der von $\vec{u} = (\vec{ r}, \dot{\vec{r}})$ aufgespannte 2n dimensionale Vektorraum heißt Phasenraum, Definiere
							\[ u_j = y_j \quad j = 1, ... , n \qquad u_{j+n} = y_j \quad j = 1,...,n \]
							$ \Rightarrow \dot{u}_j = \dot{y}_i  = u_{n+j} \qquad j=1,...,n$\\
							$ \dot{u}_{j+n} = \ddot{y}_j  = f_j(u,...,\underbrace{u_{2n}}_{\text{ Kraft kann auch von Geschwindigkeiten abhängen, z.B. Reibung, Lorentz-Kraft}},t)$
							hat die Form:\\
							$ \dot{\vec{u}} = \vec{g}(\vec{u},t) \rightarrow \text{2n gewöhnliche Differentialgleichung \textbf{erster} Ordnung}$\\
							Symmetrien dieser Gleichungen führen i.a. zu Erhaltungsgrößem z.B. führt Zeit-unabhängigkeit i.a. zu Energieerhaltung\\
							$\vec{ F}$ heißt Gradienten-Kraft wenn
							\begin{center}
								\includegraphics[scale=0.4]{Jetsteam_gradientkraft_vom_aequator_zum_pol_1.png}
							\end{center}
							\[ \vec{ F}(\vec{r}) 0 - \vec{\nabla} v (\vec{ r}) \]
							mit $V(\vec{ r})$ eine Skalarfunktion genannt \textbf{\textit{Potential}}\\
							Dann ist
							\[ e = E_{kin} + E_{pot} = \frac{m}{2} \dot{\vec{ r}}^2 +V(\vec{r}) \]
							erhalten:
							$ \frac{dE}{dt} = m \dot{\vec{ r}}, \ddot{\vec{ r}} + \dot{\vec{ r}} * \vec{\nabla} v = \dot{\vec{ r}} * (m\ddot{\vec{ r}} + \vec{\nabla} v) \\
							 \dot{\vec{ r}} * (m\ddot{\vec{ r}} - F(\vec{ r}) = 0)$
							 \subsubsection{Beispiele}
								Schräger Wurf $\rightarrow$ siehe experimenteller Teil\\
								mathematisches Pendel (Kräfte Gleichsetzten)
								\begin{figure}[htbp]
								 	\begin{minipage}[t]{10cm}
									\vspace{0pt}
									\centering
									\includegraphics[scale=0.5]{Mathependell.png}
									\end{minipage}
									\hfill
									\begin{minipage}[t]{10cm}
										\vspace{0pt}
										$\vec{ F} = -mg \vec{e}_z$\\
										$= \vec{\nabla} E_{pot}$\\
										mit $ E_{pot} = mgh = mgl(1-cos\varphi)  $
									\end{minipage}
								\end{figure}
							wenn $l = const$ ist $F_\varphi = m a_\varphi$ in Polarkoordinaten\\
							$F_\varphi = - mg sin \varphi \qquad a_\varphi = l \ddot{\varphi}$\\
							$\Rightarrow ml\ddot{\varphi} = -mg sin \varphi$ oder $ \ddot{\varphi} + \frac{g}{e} sin \varphi = 0 $
						\subsection{Energie-Erhaltung}
							\quad$ E = E_{kin} + E_{pot} = \frac{m}{2} e^2 \dot{\varphi}^2 + mgl(1 - cos \varphi)= const = mgl(1 - cos \varphi_0) $\\
							$ \varphi_0 =$ maximale Auslenkung; Falls $\varphi_0 \leq \pi$\\
							\[ \Rightarrow \dot{\varphi} = \sqrt{\frac{2g}{l}(cos \varphi - cos \varphi_0) } \]
							damit kann man die Schwingungsperiode berechnen:\\
							$T = \int_{0}^{T}dt = 4 \int_{0}^{T/4} = 4 \int_{0}^{0} \frac{d\varphi}{\dot{\varphi}} = 4 \sqrt{\frac{l}{2g}} \int_{0}^{0} \frac{d\varphi}{\sqrt{cos \varphi - cos \varphi_0}} $\\
							Verwende 
							\[ cos \varphi - cos \varphi_0 = 2(sin^2\frac{\varphi_0}{2} - sin^2 \frac{\varphi}{2}) \]
							definiere $h = sin^2 \frac{\varphi_0}{2}$ und trnsformiere auf neue Variable $\xi$ mit  sin $\frac{\frac{\varphi}{2} = \sqrt{2} sin \xi}{\varphi_0} so daß 0 \leq \xi \leq \frac{\pi}{2}$
							\[ \Rightarrow \frac{1}{\sqrt{2}} \int_{0}^{\varphi_0} \frac{d\varphi}{\sqrt{cos \varphi - cos \varphi_0}} = \frac{1}{2} \int_{0}^{\pi / 2} \frac{d \varphi}{d \xi} \frac{d \xi}{\sqrt{h - h sin^2\xi}} \]
							dann ist $-\frac{1}{2} cos \frac{\varphi}{2} d\varphi = \sqrt{h} cos \xi d \xi \Rightarrow \frac{d \varphi}{d \xi} = 2 \sqrt{h} \frac{cos \xi}{cos \frac{/varphi}{2}}$\\
							$ \Rightarrow \frac{1}{\sqrt{2}} \int_{0}^{\varphi_0} \frac{d \varphi}{\sqrt{cos \varphi - cos \varphi_0}} = \frac{1}{2} \int_{0}^{\pi/2} \frac{2 \sqrt{h} cos \xi d\xi}{\sqrt{h} cos \xi cos \frac{\varphi}{2}} = \int_{0}^{\frac{\pi}{2}} \frac{d \xi}{\sqrt{1 - sin^2 \frac{\varphi}{2}}} = \frac{d \xi}{\sqrt{1 -h sin^2\xi}} = K(h)$ = vollständiges elliptisches Integral \\
							$ \Rightarrow T= 4 \sqrt{\frac{l}{g}} K(sin^2 \frac{\varphi_0}{2}) $\\
							\textbf{Taylor Entwicklung von K(h) um k = 0:}\\
								$K(h) \approx \int_{0}^{\frac{\pi}{2}} d\xi (^1 + \frac{h}{2} sin^2 \xi) \approx \frac{\pi}{2}(1+\frac{h}{4})$\\
								$ \Rightarrow T = 2\pi \sqrt{\frac{l}{g}}(1 +  \frac{h}{4}) $\\
								Für $k \rightarrow 0 ist T \rightarrow 2\pi \sqrt{\frac{l}{g}}$ Das entspricht der harmonischen Nährung $sin \varphi \approx \varphi$:\\
								$\ddot{\varphi} + \frac{g}{l} sin \varphi \approx \ddot{\varphi} + \frac{g}{l} \varphi = 0 $\\
								\textbf{Lösungen sind dafür:}
									\[ \varphi(t) = A sin w_0 t + B cos w_0 t \]
									mit $w_0 =\sqrt{\frac{g}{l}}$,aber $T = \frac{2 \pi}{w_0} = 2\pi \sqrt{\frac{l}{g}}$\\
									Dies entspricht dem Federpendel
									\[ m \ddot{x} + hk = 0 \text{ mit } E_{pot} = \frac{h}{2}x^2 \quad F = - \frac{d E_{pot}}{dx} = - hx \] 
							\part{Vorlesung 11}
								\subsection{Erhaltung von Impuls und Drehimpuls}
									Impuls eines Teilchens := $\vec{p} = m \vec{ v} = m \dot{\vec{r}}$\\
									Dabei ist m eine Ruhemasse in Newtonscher Physik ansonsten $m \rightarrow m \gamma \text{\textit{ bzw. bewegte Masse}}$\\
									$ \Rightarrow \dot{\vec{p}} = \vec{ F}(\vec{r},t) $\\
										\[ \text{Drehimpuls } \vec{L}:= m\vec{ r} \times \dot{\vec{ r}} = \vec{ r} \times \vec{p} \]
								$\Rightarrow \dot{\vec{L}} = m \dot {\vec{r} \times \dot{\vec{r}}}_{= 0} + m \vec{ r} \times \ddot{\vec{r}} = \vec{r} \times \dot{\vec{p}} = \vec{ r} \times \vec{ F}$\\
								Dieses Ergebnis der Umformung am Ende ist auch bekannt als Drehmoment.
								\subsubsection{N-Teilchensystem: }
								$r_{jk} := |\vec{r}_j - \vec{r}_{k}| = r_{kj} $\\
								Angemommen nur Zentralkräfte $\vec{F}_{jk}$ wirken von Teilchen $k$ auf Teilchen $j$.\\
								$\Rightarrow \vec{ F}_{jk} = - \vec{F}_{jk}(r_{jk}) \frac{\vec{ r}_j - \vec{r}_k}{r_{jk}} $ \quad also $\dot{\vec{r}_j} = \sum_{k=1\\ k \neq j }^{N} \vec{ F}_{jk}	$\\
								insbesondere 
								\[ \vec{ F}_{jk} = - \vec{ F}_{kj} \rightarrow \text{3. Newton$^\prime$ sche Axiom!} \]
								Definiere:
								\[ \vec{p} = \sum_{j} \dot{\vec{p}}_j  = \sum_{j \neq k} \vec{ F}_{jk} = 0  \]
								$ \Rightarrow  \text{ Schwerpunkt } \vec{R} = \frac{I}{M} \sum_{\hat{j} = 1}^{N} m_j \vec{r}_j \text{ bewegt sich mit Geschwindigkeit	} $\\
								\[ \vec{ v}= \dot{\vec{R}} = \frac{I}{M} \sum_{\hat{j} = 1}^{N} m_j  \vec{r}_j = \frac{1}{M} \sum \vec{p}_j = \frac{\vec{p}}{M} = const \]
								Damit ist der Schwerpunkt $ \vec{R} =$
								\[ \dot{\vec{L}} = \sum_{j}  \dot{\vec{L}}_j  = \sum_{j} r \vec{ r}_j \times \sum_{k, k \neq j} \vec{F}_{j k} = \sum_{j \neq k} \frac{F_{jk} (r_{jk} } {r_{jk}} \vec{ r}_j \times (\vec{ r}_j - \vec{ r}_k) \]
								\[ = - \sigma_{\hat{j} \neq k } \frac{F_{jk} (r_{jk}) }{r_{jk}} \vec{r}_j \times \vec{ r}_k = 0 \text{ weil } \vec{ r}_j \times \vec{ r}_k = - \vec{ r}_k \times \vec{ r}_j \]
							\subsection{Zweiteilchensystem}
								\textbf{Definiere:}\\
								\[ \vec{ r} = \vec{ r}_{12} = \vec{ r}_1 - \vec{ r}_2 = \text{ Relativkoordinate } \]
							$ M = m_1 + m_2 = \text{ Gesamtmasse}  \quad M = \frac{m_1 m_2}{m_1+ m_2} = \text{ steht für die reduzierte Masse}$\\
							\[ \vec{R} = \frac{1}{M}(m_1 \vec{ r}_i + m_2 \vec{ r}_2) \text{Steht für den Schwerpunkt} \]
							Damit gilt
							\[ \vec{F}(\vec{r}) :=  \underbrace{\vec{F}_{12} }_{\text{ Zentralkraft }} = F(r) = F(r) \hat{\vec{r}} \]
							Nun können wir eine Bewegungsgleichung aufstellen. Dies tuen wir indem wir einfach die normale Bewegungsgleichung für Zweiteilchensysteme nehmen.
							\[ m_1 \ddot{r}_1 = \vec{F}_{12} \qquad m_2 \ddot{r}_2 = \vec{F}_{21}  \]
							$ \Rightarrow = 0 \text{ wie im N-Körpersystem } $\\
							\[ \ddot{\vec{r}} =\ddot{\vec{ r}}_1 - \ddot{\vec{ r }}_2 = \frac{1}{m_1} \vec{ F}_{12} - \frac{1}{m_2} \vec{F}_{21} = \left( \frac{1}{m_1} + \frac{1}{m_2} \right) \vec{F}(\vec{r}) \]
							$ \Rightarrow \mu \ddot{\vec{r}}_1	 = \vec{F}(\vec{r}) $\\
							Ferner:\\
							\[ \vec{'L}_{total} = \vec{L} _1 + \vec{L}_2 = \vec{L}_s + \vec{L}\]
							wobei 
							\[ \vec{L}_s = M \vec{R}\times \dot{\vec{R}} = \vec{R} \times \vec{p} = \text{ Drehimpuls der Schwerpunktbewegung} \]
							und 
							\[ \vec{L} = \vec{r} \times \vec{p} = \mu \vec{r} \times  \dot{\vec{r}} = \text{ relativ Drehimpuls} \]
						\section{Stoßprobleme}
							Eine Masse $m_1$ mit Geschwindigkeit $\vec{u}_1$ Stoße auf eine Masse $m_2$ in Ruhe\\
							Dann gilt:
								\[\frac{1}{2}m_1u_1^2 = \frac{1}{2} m_1 v_1^2 + \frac{1}{2} m_2 v_2^2 \quad \text{ Energieerhaltung} (1 \]
								\[ m_1 \vec{u}_1 = m_1 \vec{v}_1 + m_2 \vec{v}_2 \qquad \text{Impulserhaltung} (2) \] 
								wobei $\vec{v}_1, \vec{v}_2$ die Endgeschwindigkeiten von $m_1$ und $m_2$ sind.\\
								Die wahl des Koordinatensystems geschieht so, daß der Stoß in $xy-Ebene$ stattfindet mit $\vec{u}_1 = u:1 \vec{e}_x, \xi := P2x = m_2 \vec{v}_2 \cdot \vec{e}_x  ;\eta := P2y = m_2 \vec{v}_2 \cdot \vec{e}_y $
							\begin{center}
								\includegraphics[scale=0.5]{Stoßproblem1.png}
							\end{center}
						\[ \xi^2 + \eta^2 = m_2^2 v_2^2 (3); \qquad (m_1 u_1 - \xi)^2 + \eta^2 = m_1^2 v_1^2 (4) \]
						Nun ist 
						\[ m_1^2 v_1^2 \underbrace{= }_{(1)} m_1^2 (u_1^2 - \frac{m_1}{m_2} v_2^2 \underbrace{=  }_{(3)} m_1^2 u_1^2 - \frac{m_1}{m_2} (\xi^2 + \eta^2) \]
						Nun kann man in $(4)$ einsetzen um zubekommen:
						\[ (\xi ^2 + \eta^2 )(1+ \frac{m_1}{m_2}) - 2m_1 u_1 \xi = 0 \]
						Multiplikation mit $\frac{m_2}{m_1 + m_2}$ gibt:
						\[ \xi^2 + \eta^2 - 2 \mu u_1 \xi = 0 \text{ oder } (\xi - \mu u_1)^2 + \eta^2 = (\mu u_1)^2 \]
						wobei $\mu \equiv \frac{m_1 m_2}{m_1 + m_2}$ die reduzierte Masse ist.
						\begin{center}
							\includegraphics[scale=0.5]{reduzierteMasse1.png}
						\end{center} 
					\part{Vorlesung 12}
						\subsection{1. zentraler Stoß:}
							Alle Impuls liegen auf der $x-Achse$ so daß $m_1 u_1 = m_1 v_1 + m_2 v_2$\\
							Anwendung der vorigen Figur ergibt (abgesehen von der trivialen Lösung $v_1 = u_1, v_2 = 0$)
						\[ m_2 v_2 = = 2 \mu u_1 \quad \Rightarrow v_2 = \frac{2m_1}{m_1 + m_2 } u_1 \]
						\[ m_1 v_1 = m_1 u_1 - m_2 v_2 = (m_1 - 2 \mu)u_1  = \frac{m_1^2 + m_1m_2 - 2m_1 m_2}{m_1 + m_2} \]
						\[] u_1 = \frac{m_1 - m_2}{m_1 + m_2} m_1 u_1 \]
						In der Tat sind Energie- und Impulserhaltung erfüllt.
						\[ \frac{1}{2} m_1 v_1^2 + \frac{1}{2} m_2 v_2^2 = \frac{m_1}{2(m_1 + m_2)^2} ((m_1  m_2 )^2 + 4m_1 m_2) u_^2 = \frac{m_1}{2} u_1^2 \]
						\[ m_1 v_1 + m_2 v_2 = \frac{1}{m_1 + m_2}(m_1^2 - m_1 m_2 + 2m_1 m_2) u_1 = m_1 u_1 \]
						\subsection{2. $ m_1 = m_2 = m $ (gleiche Massen)}
						$ \Rightarrow \mu = \frac{m}{2} \quad \mu u:1 = \frac{m}{2} u_1 = \frac{m_1 u_1}{2} \Rightarrow \vec{v}_1 \perp  \vec{v}_2 $\\
						für einen zentralen Stoß gilt damit $v_2 = u_1 , v_1 = 0$\\
						$ \rightarrow$ stoßender Körper kommt zur Ruhe $\rightarrow$ z.B. Billardkugel
						
						\subsection{$ m_1 >> m_2 $}
						$ \Rightarrow $
						\[  \mu \approx m_2 \quad \mu u_1 \approxm_2 u_1 << m_1 u_1 \]
						für einen zentralen Stoß hat man $ m_2v_2 \approx2m_2 u_1 \Rightarrow v_2 \approx2 u_1 $\\
						$ m_1 v_1 \approx m_1 u_1 - m_2 2u_1 \Rightarrowv_1 \approx m_1 u_1$ \\
						Die Energieübertrag ist dann:
						\[ \frac{\frac{1}{2} m_2 v_2^2 }{\frac{1}{2} m_1 7_1 ^2} \approx 4 \frac{m_2}{m_1} << 1 \]
						\begin{center}
							\includegraphics[scale=0.3]{gleichemassenstoß.png}
						\end{center}
						
					\subsection{4. $m_1 << m_2}$
						$\Rightarrow \mu \approx m_1 \qquad \mu u_1 \approx m_1 u_1$\\
						\[ m_2v_2 \approx 2 m_1 u_1 \Rightarrow v_2 \approx  2 \frac{m_1}{m_2} u_1 \]
						\[ m_1 v_1 \approx m_1u_1 - 2m_1 u_1 \Rightarrow v_1 \Rightarrow v_1 \approx - u_1 \Rightarrow \text{ Reflextion mit Impulsübertrag} \]
						also 
						\[ \delta p = - 2 m_1 u:1 \right \text{ wichtig in der hinetischen \textbf{Gastheorie}} \]
						\[ \text{Energieübertrag = } \frac{\frac{1}{2} m_2 v_2^2}{\frac{1}{2} m_1 u_1^2} \approx 4 \frac{m_1}{m_2} << 1 \]
						\begin{center}
							\includegraphics[scale=]{mm.png}
						\end{center}
					\subsection{Zentraler inelastischer Stoß zweier gleicher Massen}
					\[ \frac{m}{2} u_^2 = \frac{m}{2} v_1^2 + \frac{m}{2} v_2^2 + \underbrace{Q}_{Wärme} \quad  (1); u_1 = v_1+ v_2  \]
					Einsetzen von $ v_1 = u_1 - v_2 $in$ (1) $ ergibt
					\begin{align*}
					v_2^2 - v_2 u_1 + \frac{\psi}{m} = 0 &\Rightarrow v_2 = \frac{u_1}{2} + \sqrt{\frac{2}_1^2{4 } - \frac{Q}{m}}\\
					&\Rightarrow v_1 = \frac{u_1}{2} - \sqrt{\frac{u_1^2}{4} - \frac{\psi}{m}}
					\end{align*}
					Aus der Wurzel folgt auch
					\[ Q \leq \frac{m}{4} u_^^2  \text{ und im Maximalfall gilt } v_1 = v_2 \frac{u_1}{2} \]
						\section{Keplerproblem}
							\subsection{Zentralkraftfelder}
								Potential $ V(\vec{r}) = V(r) $\\
							$\vec{F}(\vec{ r }) = - \vec{\nabla} V(r) = - \frac{dV}{dr} \vec{e}_r \qquad \vec{e}_r  = \frac{\vec{ r}}{r} $\\
							Erhaltungsgrößen:
							\[ F = E_{kin} + E_{pot} = \frac{m}{2} (\dot{\vec{ r }})^2 + E(\vec{ r }) \] ist dann die Energie
							\[ \vec{L} = \vec{ r }\times \vec{p} = m \vec{ r } \times \dot{\vec{ r}} \]
								\[ \dot{\vec{L}} = m \dot{\vec{ r }} \times \dot{\vec{ r }} + m \vec{ r } \times \ddot{\vec{ r }} = \vec{0} = 0 \]
							Die Newtonsche Kraft entspricht dann: $ m \ddot{\vec{ r }}  = \vec{ F} (\vec{ r }) = F(r)\vec{e}_r$\\
							Ebene Polarkoordinaten: $ \dot{\vec{ r }}  = \dot{r} \vec{e}_r + r \dot{\varphi} \vec{e}_\varphi $ 
							\begin{align*}
								\Rightarrow E = \frac{m}{2}(\dot{r} \vec{e}_r + r \dot{\varphi} \vec{e}_\varphi) ^2 + E_{pot} (r) = \frac{m}{2}(\dot{r}^2+ r^2 \dot{\varphi}^2) + E_{pot}(r)\\
								\vec{L} = m \vec{r}x (\dot{r\vec{e}_r} + r \dot{\varphi} \vec{e}_r) = mr\dot{\varphi}\vec{r} \times \vec{e}_\varphi \underbrace{= }_{\vec{r} = r \vec{e}_r} mr^2 \dot{\varphi} \underbrace{\vec{e}_r \times \vec{e}_\varphi}_{\vec{e}_z} = mr^2 \dot{\varphi} \vec{e}_z\\
								\vec{L}^2 = L^2 = (mr^2 \dot{\varphi})^2 = const \Rightarrow mr^2\dot{\varphi} = const = L \Rightarrow \frac{L}{mr^2}\\
								\Rightarrow E = \frac{m}{2} \dot{r}^2 + \frac{L^2}{2mr^2} + E_{pot} (r)= \frac{m}{2} \dot{r}^2 + V_{eff}(r)\\
								V_{eff}(r) = E_{pot}(r) + \underbrace{\frac{L^2}{2mr^2}}_{\text{Drehimpulsbarriere}}\\
								E= const \Rightarrow \dot{r} = \frac{dr}{dt} = \pm \sqrt{\frac{2}{m} (E - V_{eff}(r))} \qquad \textbf{(1)}\\
								\Rightarrow \dot{\varphi} = \frac{L}{mr^2} \qquad \textbf{(2)}
							\end{align*}
						$t(r) - t(r_0) = \int_{r}^{r_0} dt^\prime = \int_{r_0}^{r} = \frac{dr^\prime}{dr^\prime / dt} \\ = \int_{r_0}^{r} \frac{dr^\prime}{\dot{r(r^\prime)}} = \int_{r_0}^{r} \frac{dr^\prime}{\sqrt{\frac{2}{m}(E - V_{eff}(r^\prime)) } }  $\\
						\textbf{Ähnlich für Azimentalwinkel:}\\
						\[ \varphi(r) - \varphi(r_0) = \int_{\varphi(r_0)}^{\varphi(r)} d\varphi^\prime = \int_{r_0}^{r} \frac{d\varphi}{dr} \frac{dt}{dr} dr^\prime = \int_{r_0}^{r} \dot{\varphi} \frac{1}{\dot{r}(r^\prime)} dr^\prime \]
					$=\frac{L}{\sqrt{2m}} \int_{r_0}^{r} \frac{dr^\prime}{r^{12} \sqrt{E \cdot V_{eff} } }$
					\[ V_{eff} (r) \leq E \]
					Im allgemeinen : \quad $\lim\limits_{r \to \infty} E_{pot} (r) = 0$\\
					$\Rightarrow$ für ungebunde Bahnen(r.B. Meteore die nicht aus unserem Sonnensystem kommen) ist $E \geq 0$\\
					Für gebundene Bahnen ist $E < 0:$ 
					\[ r_{max} \geq r \geq r_{min} \]
					gebundene Bahnen können geschlossen aber ungeschlossen sein.\\\\
					Für geschlossene Bahnen muß:
					\[ \Delta \varphi = \frac{2L}{\sqrt{2m}} \int_{r_{min}}^{r_{max}} \frac{dr^\prime}{r^{12} \sqrt{E - V_{eff} (r^\prime)}} = 2 \pi n \]
				\part{Vorlesung 13}
					\subsubsection{Beispiele für potentielle Energie}
						\begin{align*}
							E_{pot} = - \frac{G_N m_m}{r} \text{ Gravitationspotenzial zwischen M und m}\\
							E_{pot} = \frac{\vec{e_1} \vec{e_2}}{4 \pi \varepsilon_0 r} \text{ elektrisches Potenzial zwischen zwei Ladungen} e_1, e_2\\\\
							E_{pot}(r) = \alpha\left[ \underbrace{\left(\frac{r_0}{r}^{12} \right)}_{Abstoßung} - \underbrace{2 \left(\frac{r_0}{r} \right)^6}_{Anziehung} \right] \text{ Lennard-Jones Potenzial aus der Kernphysik}
						\end{align*}
					\subsection{Keplerproblem}
					zwei Massen m,M \qquad  i.a. $M \rightarrow m$\\
					wähle Koordinatensystem so daß der Schwerpunkt sich am Ursprung und in Ruhe befindet
					\[ \vec{R} = \frac{M \vec{r}, m \vec{r}_2 }{M + m}  = 0 \Rightarrow \vec{r}_1 = - \frac{m \vec{r}_2}{M} r.B. \frac{m}{M}  << 1 \]
				\[ \vec{r}  := \vec{r}_2 - \vec{r}_1 = \frac{M + m}{M}\vec{r}_2  \Rightarrow \vec{r}_2 = \frac{M}{M + m} \vec{r} \quad \vec{r}_1 = \frac{m}{M + m} \]
				\[ E_{pot}(r) = \frac{G_N (M+m) \mu }{r} \quad \mu = \frac{Mm}{M + m} \text{
				reduzierte Masse} \]
			$\Rightarrow$
			\[ E = \frac{\mu}{2} \dot{r}^2 + V_{eff} = const. \quad V_{eff}(r) = - \frac{GMm}{r} + \frac{L^2}{2\mu r^2}  \]
			\[ \vec{L} = \mu \vec{r} \times \dot{\vec{r}} = const \]
			Für ein $\frac{1}{r}$ Potential ist auch der Lenz-Russe-Vektor erhalten. Dies muss natürlich erstmal beweisen werden wofür wir die Vektoridentitäten benötigen.
			\[ \vec{A} = \dot{\vec{r}} \times \vec{L} - \frac{G_N Mm}{r}\vec{r} \]
			damit erhalten wir:
		\[	\vec{A} = \ddot{\vec{r}} \times \vec{L} + \frac{G_N Mm}{r^2} \dot{r} \vec{r} - \frac{G_N}{M} \dot{\vec{r}}\]
		\[	= \frac{G_NMm}{r} \left( - \frac{\vec{r}}{\mu r^2} \times (\mu \vec{r} \times \dot{\vec{r}}) + \frac{\dot{r}}{r} \vec{r} - \dot{\vec{r}} \right)\]
		\[	\text{wobei dies genutzt wird }\mu \ddot{\vec{r}} = \vec{F}(\vec{r})- \vec{\nabla} V (r) = - \frac{G_N Mm}{r^3} \vec{r}\]
			\[= \frac{G_N Mm}{r} \left(\dot{\vec{r}} - \frac{\dot{r}}{r} \vec{r} + \frac{\dot{r}}{r} \vec{r} - \dot{\vec{r}}  \right) = 0\]
			wobei wieder gilt
			\[\vec{r} \times (\vec{r} \times \dot{\vec{r}}) \underbrace{=}_{bac-cab-Regel} \vec{r} (\vec{r} \cdot \dot{\vec{r}}) - \dot{\vec{r}} (\vec{r} \cdot \vec{r}) =\underbrace{=}_{\frac{d}{dt}(r^2 ) = 2 r \dot{r} = \frac{d}{dt} ( \vec{r} \cdot \vec{r}) = 2\vec{r} \cdot \dot{\vec{r}}} \vec{r} r \dot{r} - r^2 \dot{\vec{r}} \]
			\qed\\
			Für den Betrag gilt dann:
			\begin{align*}
		A^2 = \vec{A} \cdot \vec{A} = (\dot{\vec{r}} \times \vec{L} - \frac{G_N Mm}{r} \vec{r})^2 \underbrace{=}_{|\dot{\vec{r}} \times \vec{L}| = \dot{\vec{r}}^2 \vec{L}^2 \text{ weil } \dot{\vec{r}} \cdot \vec{L} = 0} \dot{\vec{r}}^2 \vec{L}^2 - \frac{2G_N Mm}{r} (\dot{\vec{r}} \times \vec{L} ) \cdot \vec{r} + (G_N Mm)^2
		\end{align*}
				\[ \underbrace{=}_{(\dot{\vec{r}} \times \vec{L}) \cdot \vec{r} =(\vec{r} \times \dot{\vec{r}}) \cdot \vec{L} = \frac{\vec{L}^2}{\mu }} L^2(\dot{\vec{r}}^2 - \frac{2G_N Mm }{\mu r}) + (G_N Mm)^2 = \frac{2L^2}{\mu} E + (G_N Mm)^2 \]
Nun definieren wir die nummerische Exzentrizität:\\
$\varepsilon := \sqrt{ 1 + \frac{2 L^2 E}{ \mu (G_N Mm)^2}} \quad (1)$
\[ \Rightarrow A = G_NMm \varepsilon \]
\begin{center}
\includegraphics[scale=0.6]{exzentri.png}
\end{center}
\[ \vec{A} \cdot \vec{r} = Ar cos \varphi = G_N Mm \varepsilon r cos \varphi \]
\[ \vec{A} \cdot \vec{r} = (\dot{\vec{r}} \times \vec{L}) \cdot \vec{r} - \frac{G_N Mm}{r} \vec{r} \cdot \vec{r} = \frac{L^2}{\sigma} - G_N Mm r \]
$\Rightarrow r(\varphi) = \frac{k}{ 1 + \varepsilon cos \varphi} \qquad k = \frac{L^2}{G_N Mm \sigma} \quad (1)$
\begin{center}
	\includegraphics[scale=0.5]{ellipseexzen.png}
\end{center}
\textbf{1.Definition:} 
\[ \frac{x^2}{a^2} + \frac{y^2}{b^2} = 1 \]
 So kann man zeigen daß dies äquivalent ist zu \\
 \textbf{Definition 2:}
\[r_1+r_2 = const = 2a = (a-e) + (a+e ) \]
 	$\Rightarrow r_1^2 = (x - e)^2 + y^2 \qquad r_2^2 = (x +e)^2 +y^2$\\
 	$\Rightarrow r_1 - r_2 = \frac{2_1^2 - r_2^2}{r_1 + r_2} = \frac{1}{2a} [ (x-e)^2+ y^2 - (x+ e)^2 - y^2 ] = - 2 \frac{e}{a} x = - 2 \varepsilon x $\\
$ \Rightarrow r_1 = \frac{1}{2} (r_1 + r_2 ) + \frac{1}{2} (r_1 - r_2 ) = a - \varepsilon x = a - \varepsilon (e + r_1 cos \varphi) $\\
$ \quad = a - \varepsilon e - \varepsilon r_1 cos \varphi = \frac{a^2 - e^2}{a} -\varepsilon r_1 cos \varphi = \frac{b^2}{a} - \ r_1 cos \varphi$\\
$ \Rightarrow r_1 = \frac{b^2 / a}{1 + \varepsilon cos \varphi} = \frac{k }{1 + \varepsilon cos \varphi} \text{ mit } k = \frac{b^2}{a} $ 
Spezialfall $e = \varepsilon = 0 \rightarrow$ Kreis\\
Drücke Halbachsen a und b durch E und L aus$:$
\[ a = \frac{1}{2} (r(\varphi = 0) + r(e= \pi )) = \frac{k}{2} (\frac{1}{1 + \varepsilon} + \frac{1}{1 - \varepsilon}) = \frac{k}{1 - \varepsilon^2}  \]
nun setzt man die oben mit $(1)$ markierten Gleichungen ein$:$
\[ \frac{L^2}{G_N Mm \mu} (- \frac{2L^2 E}{(G_N Mm)^2 \mu} = - \frac{G_N Mm}{2E} > 0 \qquad (3) \]
$ = \frac{L}{\sqrt{-2 \mu E}}  \qquad (4) $
Er gilt der Flächensatz$:$ Der Radiusvektor überstreicht in gleichen Zeiten gleiche Fläche (Zweiter Keplerscheres Gesetz)
	\begin{figure}[htbp]
	\begin{minipage}[t]{10cm}
		\vspace{0pt}
		\centering
		\includegraphics[scale=0.5]{kepler2.png}
	\end{minipage}
	\hfill
	\begin{minipage}[t]{10cm}
		\vspace{0pt}
		$ d\vec{F} = \frac{1}{2} \vec{r \times d\vec{r}} $\\
		$ \Rightarrow \frac{d\vec{F}}{dt} = \frac{1}{2} \vec{r} \times \dot{\vec{r}} = \frac{\vec{L}}{2\mu } = const $
	\end{minipage}
\end{figure}
	\[ F = \pi ab = \pi a \frac{L}{\sqrt{-2 \mu E}} \]
	anderseits $ F = \int\limits_{T}^{0} \frac{dF}{dt} dt = \frac{1}{2\mu} \int\limits_{T}^{0} L dt = \frac{LT}{2 \mu} $
	\[ \Rightarrow T = \pi a \sqrt{ \frac{2\mu }{-E}} \underbrace{=}_{(3): elemeniere E} 2\pi a \sqrt{ \frac{\mu a}{G_N Mm}} \]
	\[ \Rightarrow \frac{T^2}{a^3} = \frac{4\pi^2 \mu }{G_N Mm} = const  = \frac{4 \pi ^2}{G_N}(M + m) \approx \frac{4 \pi^2}{G_N M} \]
	$\rightarrow$ \textbf{drittes Keplersches Gesetz}
	\subsection{2. Hyperbel}
	\begin{center}
		\includegraphics[scale=0.5]{Hyperbelproblem.png}
	\end{center}
ohne detaillierte Beweise:\\
\textbf{1- Definition}\\
\[ \frac{x^2}{a^2} - \frac{y^2}{b^2} = 1 \]
\textbf{2. Definition:}\\
\[ |r_1 - r_\mathds{1} = 2a \]
\begin{align*}
	\Rightarrow r_1 \frac{k}{1 + \varepsilon cos \varphi} &\qquad k = \frac{b^2}{a}\\
	a= \frac{G_N}Mm{2E} > 0&\qquad b = \frac{L}{\sqrt{2\mu E}} 
\end{align*}

	\begin{figure}[htbp]
	\begin{minipage}[t]{10cm}
		\vspace{0pt}
		\centering
		\includegraphics[scale=0.5]{Hyperbel1.png}
	\end{minipage}
	\hfill
	\begin{minipage}[t]{10cm}
		\vspace{0pt}
		Streuung:\\
		$ cos \varphi_\infty = - \frac{1}{\varepsilon}$\\
		$ \theta = 2 \varphi_\infty -\pi \rightarrow $ Streuwinkel\\
		$ 0 \leq \theta \leq \pi  
	\end{minipage}
\end{figure}
$ \Rightarrow$ sin \frac{ \theta}{2} = sin (\varphi_\infty - \frac{ \pi}{2}) = - cos \varphi_\infty = \frac{1}{\varepsilon}$\\
$ \Rightarrow tan \frac{\theta}{2} = \frac{sin \frac{\theta}{2}}{\sqrt{1- sin \frac{\theta}{2}}} = (sin ^{-2} \frac{\theta}{2} -1)^{- \frac{1}{2}} $ \\
$ (\frac{e^2}{a^2} -1)^{-\frac{1}{2}} = (\frac{b^2}{a^2})^{-\frac{1}{2}} = \frac{a}{b} = \frac{G_N Mm}{2b E}  $
\[ \Rightarrow \theta(b) = = 2 arctan \frac{G_n Mm}{2 b E} \]
b ist minimaler Abstand der Asymtote vom Kraftzentrum $=$ \textit{Stopparamer}
\subsection{3. Parabel}
$\rightarrow$ siehe frühere Übungsaufgabe\\
\textbf{Anwendung:} Keplerbahnen sind geschlossen; wenn das Potential nicht $\alpha \cdot \frac{1}{r}$ wie z.B. im allgemeiner Relativitätstheorie, dann sind Bahnen nicht geschlossen\\
z.B. Perihelddrehung des Merkurs
		\begin{figure}[htbp]
		\begin{minipage}[t]{10cm}
			\vspace{0pt}
			\centering
			\includegraphics[scale=0.5]{Parabel.png}
		\end{minipage}
		\hfill
		\begin{minipage}[t]{10cm}
			\vspace{0pt}
			Streuung:\\
			Menge aller Punkte, die vom Brennpunkt F und \\einer Graden G den gleichen Abstand haben.
			\end{minipage}
			\end{figure}
		\[ x + \frac{k}{2} = \sqrt{y^2 + (x- \frac{k}{2})^2}   \Leftrightarrow (x + \frac{k}{2})^2 = y^2+(x - \frac{k}{2})^2 \]
		\[ \Leftrightarrow r(\varphi) = y^2 = 2 kx \]
		In Polarkoordinaten wäre das:
		\[ r(\varphi) = x + \frac{k}{2} = \frac{k}{2} - r cos \varphi + \frac{k}{2} = k - r cos \varphi \]
		\[ \Rightarrow r (\varphi) = \frac{k}{1 + cos \varphi} \]
	\subsection{Formulierte Keplergesetzt}
		\begin{enumerate}
			\item Die Planeten bewegen sich auf Ellipsen, in deren Brennpunkt die Sonne steht
			\item Der Radiusvektor (Fahrstrahl) von der Sonne zum Planeten überstreicht in gleichen Zeiten gleiche Fläche
			\item Die Quadrate der Umlaufzeiten der Planeten verhalten sich wie die Kuben ihrer großen Halbachsen $ \frac{T^2}{a^3} = const, \frac{4 \pi^2}{G_N (M + m)} $
		\end{enumerate}
	\part{Vorlesung 14}
\section{Wirkungsquerschnitt und Steuung}
	Stromdichte $ j = \Box $ einfallende Teulchen pro Zeit und Fläche Raumwinkelelement 
	\[ DU = sin \theta d \theta d \varphi \]
	Der Wirkungsquerschnitt ist dann definiert durch:
	\[ d \sigma = \frac{d \sigma}{dt \eta} d\eta  = \Box \text{ Teilchen getrennt in d $\eta$ pro Zeit} \]
	\[ =\frac{j b d b d \varphi}{j}  = b d b d \varphi \quad \text{"impact paramer" } b \]
	\begin{center}
		\includegraphics[scale=0.5]{winkelquerschnitt.png}
	\end{center}
\[ \Rightarrow \frac{d \sigma }{d \eta} = \frac{b d b d \varphi}{sin \theta d \theta d \varphi}  = \frac{b}{sin \theta} \frac{1}{sin^2 \frac{\theta}{2} }  \]
Strukturwinkel $\theta $ wird mit wachsendem impace b kleiner.
\[ \frac{db}{d\theta} < 0 \Rightarrow \frac{d \sigma }{d \eta}  = \frac{b}{sin \theta} |\frac{db}{d \theta}| \]
hier: $b ''= \frac{G_N Mm}{2E} cot \frac{\theta}{2} \Rightarrow \frac{db}{d\theta} = - \frac{G_N Mm}{4E} \frac{1}{sin^2 \frac{\theta}{2}} $ 
mit $wt \frac{ \theta}{2} = \frac{cos \frac{\theta}{2}}{sin \frac{\theta}{2}} $ und $sin \theta = 2 sin \frac{\theta}{2} cos \frac{\theta}{2} $\\
$\Rightarrow\frac{dv}{du} = \frac{(G_N Mm)^2}{IGE^2} \frac{1}{sin^4 \frac{\theta}{2}}$ \\
Rutherfordsches Gesetz (im Schwerpunktsystem) = coulomb Streuung (Rutherfordsches Atommodell)
\[ \theta für: \frac{dv}{du}  cx  \frac{1}{E^2} \]
\[ E für \frac{dv}{du} cx \frac{1}{\theta^4} für \theta << 1 \]

\section{Matrizen und Tensoren}
msn-Matrix A = $ \left(\begin{array}{ccccc}
a_{11} & a_{12} & ... & a_{1j} & a_{1n} \\ 
a_{21} & a_{22} & ... & a_{2j} & a_{2n} \\ 
... & ... & ... & ... & ... \\ 
a_{i1} & a_{i2} & ... & a_{ij} & a_{in} \\ 
a_{m1} & a_{m2} & ... & a_{mj} & a_{mn}
\end{array}\right)  \rightarrow i-te Zeile $\\
j-te Spalte 
\[ A = (a_{ij}) \]
Rechenoptionen:
\begin{itemize}
	\item Addition \qquad $C = A + B \qquad c_{ij} = a_{ij} + b_{ij}$ nur gleichartige mxn-Matrizen können addiert werden
	\item skalare Multiplikation \[ C = \lambda A  \quad c_{ij} = \lambda a_{ij} \]
	\item Matrix-Multiplikation:\\
	Sei $ A = (a_{ij}) = mxn-Matix , B = (b_{kl}) = nxr-Matrix  $\\
	$ \Rightarrow C = AB \quad c_{ij} = \sum_{k=1}^{n} a_{ik} b_{kj} = \text{mxr-Matrix} $
\end{itemize}
\textbf{Spezialfall:} $ B = x = nxl-Matrix  = $ Vektor mit n -komponenten\\
$ \qquad \Rightarrow Ax= mxl-Matrix = $ Vektor mit m Komponenten\\
z.B. $m = n \rightarrow$  quadratische Matrix 
 \[ A\vec{x} = \vec{b}  \quad  \left(\begin{array}{cccc}
 a_{11} & a_{12} & ... & a_{1n} \\ 
 a_{21} & a_{22} & ... & a_{2n} \\ 
 ... & ... & ... & ... \\ 
 a_{n2} & a_{n1} & ... & a_{nn}
 \end{array}\right)   \left( \begin{array}{c}
 x_1 \\ 
 x_2 \\ 
 ... \\ 
 x_n
 \end{array}  \right) =  \left( \begin{array}{c}
 b_1 \\ 
 b_2 \\ 
 ... \\ 
 b_n
 \end{array}  \right) \]
 Darauf entsteht dann ein Lineares Gleichungssystem:
 \[ a_{11} x_1 + a_{12} x_2 + ... a_{1n} x_n = b_1 \]
 \[ a_{21} x_1 + a_{22} x_2 + ... a_{2n} x_n = b_2 \]
 \[ ... \]
 \[ a_{n1} x_1 + a_{n2} x_2 + ... a_{nn} x_n = b_n \]
 Rechenregeln:
 \[ A + B = B+A \qquad \text{kommutativ} \]
 \[ A(B+C) = AB + AC \qquad \text{ assozitiv} \]
 \[ A(BC) = (AB)C = ABC \qquad \text{distributiv} \]
 aber: 
 \[ AB \neq BA \text{ im allgemeinen } \]
 \[ [A,B] := AB- BA \qquad \text{Kommutator} \]
 Spielt wichtige Rolle in der Quantenmechanik\\
 Die $"$Transponierte$" A^T$ einer Matrix A erhält man durch Vertauschen von Zeilen und Spalten: $ (AT)_{ij} = A_{ji} $\\
 Es gilt: $ (AB)^T = B^T A^T $
 \part{Vorlesung 15}
\subsection{Quadratische nxn Matrizen}
	bilden einer $"$Algebra$"$ unter Addition und Multiplikation mit Zahlen sowie Matrixmultiplikationen.\\
	Einheitsmatrix:
	\[ E = \mathds{1} (\delta_{ij}) = \left( \begin{array}{cccc}
	1 & 0 & ... & 0 \\ 
	0 & 1 & ... & 0 \\ 
	... & ... & ... & ... \\ 
	0 & .... & ... & 1
	\end{array}  \right) \]
	\[ EA = EAE = A \forall A\]
Die Determinante ist damit:
\[ |A| = det A =\underbrace{ \sum}_{\text{Permutation} P(1,2,...,n)} (-1^P a_{1P1} a_{2P2} ... a_{ipi} ... a_{npn} \]
\[( -1 ^P  = \begin{cases}
1  \text{ grade Permuatitionen  }\\
-1 \text{ ungrade Permuationen}
\end{cases}\]
$\textbf{n=2}$	
\[\left| \begin{array}{cc}
a_{11} & a_{12} \\ 
a_{21} & a_{22}
\end{array}  \right|  = a_{11} a_{22} - a_{12} a_{21} \]
$\textbf{n = 3}$
\[ \left| \begin{array}{ccc}
a_{11} & a_{12} & a_{13} \\ 
a_{21} & a_{22} & a_{23} \\ 
a_{31} & a_{32} & a_{33}
\end{array}  \right| = a_{11} a_{22} a_{33} + a_{12} a_{23} a_{31} + a_{13} a_{21} a_{32} - a_{11} a_{23} a_{32}  - a_{11} a_{23} a_{32} - a_{12} a_{21} a_{33} - a_{13} a_{22} a_{31} \]
Man kann zeigen:
\[|AT| = |A| \]
\[ |AB| = |BA| = |A||B| \]
Wenn $|A| \neq 0 $ dann gibt es eine invere Matrix $A^{-1}$ mit
\[ A^{-1} A = AA^{-1} = \mathds{1} = E\]
In diesem Fall hat ein lineares Gleichungsystem 
\[ A \vec{x} = \vec{b} \text{ die eindeutige Lösung } \vec{x} = A^{-1} \vec{b} \]
Anwendung: Taylorentwicklung im $\mathbb{R}^n$
\[ f(\vec{r}) = f(\vec{r}_0) + \sum_{i = 1}^{n} \frac{\partial f }{\partial x_i} (\vec{r}_0) (x_i - x_{0i}  \]
\[ + \frac{1}{2} \sum_{i,j = 1}^{n} \frac{\partial^2 f}{\partial x_i \partial x_j } (\vec{r}_0) (x_j - x_{0i}) (x_j - x_{oj})+...  \]
\[ \vec{r} = (x_i) \qquad \vec{r}_0 = (x_{0i}) \]
kann geschrieben werden als
\[ f(\vec{r}) = f(\vec{r}_0) + \vec{\nabla} f (\vec{r}_0) \cdot (\vec{r} - \vec{r}_0) + \frac{1}{2} (\vec{r} - \vec{r}_0)^T Q (\vec{r} - \vec{r}_0)   \]
mit $ Q_{ij} = \frac{\partial^2 f}{\partial x_i \partial x_j} (\vec{r}_0) $
Beipsiel: Entwicklung des Potentials $\frac{1}{| \vec{r} - \vec{a} |}$ einer Punktquelle bei $\vec{a}$ um $\vec{r} = 0$
\[ \frac{\partial}{ \partial x_i} \frac{1}{| \vec{r} - \vec{a} |} = - \frac{1}{| \vec{r}- \vec{a} |^2} \frac{\partial}{\partial x_i} | \vec{r}-\vec{a}| = . \frac{x_i - a_i}{| \vec{r}-\vec{a} |^3}  \]
\[ \frac{\partial^2}{\partial_{xi} \partial_{xj}} \frac{1}{| \vec{r}-\vec{a} |} = - \frac{\delta_{ij}}{| \vec{r}-\vec{a} |^3 } + \frac{3(x_i - a_i) (x_j - a_j)}{| \vec{r}-\vec{a} |^5} \]
\[ \Rightarrow \frac{1}{| \vec{r}-\vec{a} |} = \frac{1}{a} + \frac{\vec{a} \cdot \vec{r}}{a^3} + \frac{\vec{r}^TQ\vec{r} }{2a^5} \]
mit $ Q_{ij} = 3 a_i a_j - a^2 \delta_{ij} $
Eigenschaften: $ Q_{ij} = Q_{ji} \text{Symmetrie} $
\[ \text{ Sput(trace) } \underbrace{Tr Q}_{\text{Summe der Diagonalelemente}} \equiv \sum_{j} Q_{jj} = \sum_{i} (3a_ja_j - a^2 \delta_{ij}) = 3a^2 - 3a^2 = 0  \]
Tensoren sind Objekte mit n Indizes die sich für jeden Index wie ein Vektor transformieren (\textit{Koordinatentransformationen siehe später})\\
Also: 
\[ \text{Skalar = Zahl = Tenser 0-ter Stufe} \]
\[ \text{ Vektor = Tensor 1-ster Stufe} \]
\[ \text{ Matrix = Tensor 2-ter Stufe } \]
\[ \varepsilon_{ij}k \text{ Tensor 3-ter Stufe } \]
$\vec{x}$ heißt \textbf{Eigenvektor} zum \textbf{Eigenvektor} $\lambda$ wenn 
\[ A\vec{x} = \lambda \vec{x} für \vec{x} \neq 0 \]
\[ \lambda \text{ Eigenwert } \Leftrightarrow |A - \lambda \mathds{1} | = 0  \]
wenn A nxn Matrix, dann ist 
\[ |A - \lambda \mathds{1} = a_0 + a_1 \lambda + ... + (-1)^n \lambda^n  = x_n (\lambda) \]
das \textbf{charakteristische Polynom} n-ten Grades mit n (u.U. auch -mehrfallnullstellen ) 
\[ x_n(\lambda  = #(\lambda_1 - \lambda)( \lambda_2 - \lambda) .. ( \lambda_n - \lambda ) \]
also sind $\lambda_i$ die Eigenwerte von A\\
\textbf{Satz:} Die Eigenwerte einer reellen symmetrischen Matrix sind reell.\\
\begin{proof}
	\[ \sum_{i,j} (a_{ij} - \lambda \delta_{ij} ) x_i x_j = 0  \]
	dabei multiplizieren multiplizieren wir: $ \sum_{j} a_{ij} x_j  = \lambda x_i$ mit $x_i^*$\\
	$ \underbrace{\Rightarrow}_{\text{komplexe Konjungation}} (a_{ij} \lambda^* \dashleftarrow_{ij}) x_i x_j^* = 0 $
	subtrahiere und verwende Symmetrie um $a_{ij}$ und $\delta_{ij}$
	\[ \Rightarrow \sum_{i,j} (-\lambda + \lambda^*) \delta_{ij} x_i^* x_j = (\lambda^* - \lambda ) \sum_{i} |\alpha_i|^2 = 0 \Rightarrow \lemda
	^* = \lambda \]
da $ | \vec{x} | \neq 0  $ 
\end{proof}
\textbf{Satz:} \quad Die Eigenvektoren einer reellen, symmetrischen Matrix zuverschiedenen Eigenvektoren sind orthogonal
\begin{proof}
	I\[ \sum_{i} a_{ij} x_j = \alpha x_i \qquad \sum_{i} a_{ij} y_j = \beta y_i \qquad \alpha \neq \beta \]
	multipliziere mit $y_i$ bzw. $x_i$ und summiere über i
	\[ \Rightarrow \sum_{ij} a_{ij} y_i x_j = \alpha \sum_{j} x_i y_i \]
	\[ \sum_{ij} a_{ij} x_i y_j = \beta \sum x_i y_i  \]
	\[ \underbrace{\Rightarrow}_{\text{Subtraktion mit} a_{ij} = a_{ji}} 0 ##= (\alpha - \beta ) \sum_{i} x_i y_i = (\alpha .  \beta) \vec{x} \cdot \vec{y} \Rightarrow \vec{x} \cdot \vec{y} = 0 \text{ da } \alpha \neq \beta \]
\end{proof}
	\subsection{Beispiel} Trägheitstensor einer Massenverteilung
	\[ \underbrace{I_{ij}}_{\text{reell und symmetrisches}} = \underbrace{\int dV}_{\text{Volumen-Integral}} \underbrace{\rho}_{\text{Massendichte}}  (\delta_{ij} r^2 - x_i x_j ) \quad \vec{r} = (x_1,x_2,x_3)  \]
	Eigenwerte sind reell und heißen Hauptträgheitsmomente mit entsprechender Hauptträgheitstensoren $( \widehat{=} $ Eigenvektoren $ \vec{x}_i  )$
	\subsection{Beispiel}\\
	asymmetrische Matrix \[A = \left( \begin{matrix} 
	1 & 4\\1 & 1
	\end{matrix} \right) \Rightarrow |A - \lambda \mathds{1} | = \left| \begin{matrix}
	1-\lambda & 4 \\ 1 & 1 - \lambda
	\end{matrix} \right| = (1-\lambda)^2 - 4 = 0 \]
	$ \Rightarrow \lambda_\pm = 3_1 - 1 $
	\[ \text{Eigenvektoren } \left( \begin{matrix}
	1-\lambda_\pm & 4 \\1 & 1-\lambda_\pm 
	\end{matrix} \right) \left( x_{\pm 1} \\ x_{\pm 2} = 0 \right)  \]
	$ \lambda = \lambda_+ \Rightarrow -2x_{+1} + 4x_{+2} = 0 \Rightarrow x_{+1} = 2x_{+2} \Rightarrow \alpha \left( \begin{matrix}
	2 \\ 1
	\end{matrix} \right) $ sind Eigenvektoren zu $ \lambda_+ = 3$ 
analog: $ \alpha \left( \begin{matrix}
-2 \\ 1
\end{matrix} \right) $  sind Eigenvektoren zu $ \lambda_- = -1  $
\subsection{Beispiel symmetrische Matrix}
\[ A = \left( \begin{matrix}
1 & 2 \\ 2 & 1  
\end{matrix} \right) \Rightarrow |a - \lambda \mathds{1} | = \left| \begin{matrix}
1- \lambda & 2 \\ 2 & 1-\lambda
\end{matrix} \right| = (1 - \lambda)^2 -4  = 0 \]
$ \Rightarrow \lambda_\pm = 3_1 - 1  $ wie vorher auch
\[ \text{Eigenvektoren:} \left( \begin{matrix}
-2 & 2 \\ 2 & -2 
\end{matrix} \right) \left( \begin{matrix}
x_{\pm 1} \\ x_{\pm 2}
\end{matrix} \right) = 0 \Rightarrow -2x_{\pm 1} + 2x_{\pm 2} = 0 \]
\[ \Rightarrow x_.{+1 } = x_{+2} \Rightarrow \alpha \left( \begin{matrix}
1 \\ 1 
\end{matrix} \right) \text{ sind Eigenvektoren zu} \lambda = \lambda_ü+ \]
\[ \left(\begin{matrix}
2 & 2 \\ 2 & 2
\end{matrix}\right) \left( \begin{matrix}
x_{-1} \\ x_{-2}
\end{matrix} \right) = 0 \Rightarrow x_{-1} = -x_{-2}  \]
$ \Rightarrow \alpha\left(\begin{matrix}
1 \\ -1
\end{matrix}\right)$ sind Eigenvektoren zu $\lambda = \lambda_-$
Die Eigenvektoren sind nun orthogonal, wie erwartet
\[ (1,1) \cdot \left( \begin{matrix}
1 \\ -1
\end{matrix} \right) = 1 - 1 = 0 \]
\section{physikalische Rolle des Trägheitstensors}
Starrer Körper rotiere mit Winkelgeschwindigkeit $ \vec{w} $ \\ Ein Massenelement $ dm = \delta dV $ am Ort $\vec{r}$ macht dann folgenden Beitrag zum Drehimpuls
\[ d\vec{L} = dm \vec{r} \times \vec{v} \underbrace{= }_}{\vec{v}= \vec{w} \times \vec{r}}  \delta  \vec{r } \times (\vec{w} \times \vec{r}) dV \underbrace{=}_{\text{bac-cab Regel}} \delta \left[ \vec{r}^2 \vec{w} - (\vec{w} \cdot \vec{r}) \vec{r} \right]  \] dV
\[ \Rightarrow DL_i = \delta \left[ r^2 w_i - x_i \sum_{j}  w_j x_j \right] dV \]
$ \Rightarrow L_i = \sum_{i} I_{ij} w_j $\\
mit dem Trägheitstensor:
\[ I_{ij} = \int DV \delta(\delta_{ij} r^2  - x_i x_j) \qquad \vec{r} = (x_{1}, x_2 , x_3 )  \]
Beitrag des Massenelements dm zur Kinetischen Energie:
\[ d E_{kin} = \frac{1}{2} dm \vec{v}^2 = \frac{1}{2} \delta (\vec{w} \times \vec{r})^2 dv \]
\[ \underbrace{= }_{(\vec{a} \times \vec{b}) \cdot ( \vec{c} \times \vec{d}) = (\vec{a} \cdot \vec{c}) ( \vec{b}\cdot \vec{d} ) - ( \vec{b} \cdot \vec{c} )  ( \vec{a \cdot \vec{d}} )  }  \]
\[ \frac{1}{2} \delta \left[ w^2 r^2 - (\vec{w} \cdot \vec{r})^2 \right] dV \]
\[ \frac{1}{2} \delta \left[ (\sum_{i} w_i^2) (\sum_{j} x_j^2) - ( \sum_{i} w_i x_j) ( \sum_{j} w_j x_j )  \right] dV \]
$ \Rightarrow E_{kin} = \frac{1}{2} \sum_{ij}  I_{ij} w_i w_j  $ mit $I_{ij}$ wieder dem Trägheitstensor
\section{Vektoren und Koordiantentransformationen}
\includegraphics[scale=0.5]{Koordinatentransformationen.png}
Drehung:
\[ \vec{e}_i^\prime = \sum_{j = 1}^{3} D_{ij} \vec{e}_j \]
Eigenschaften:
\[ (1) cos \varphi_{ik} = \vec{e}_k \cdot \vec{e}_i^\prime = \vec{e}_k \cdot \sum_{j} D_{ij} \vec{e}_j  = D_{jk} \]
\[ (2) \delta_{ik} = \vec{e}_i^\prime \cdot \vec{e}_k^\prime = (\sum_{j} D_{ij} \vec{e}_j) \cdot (\sum_{l} D_{kl} \vec{e}_l ) = \sum_{jl} D_{ij} D_{kl} \underbrace{\vec{e}_j \cdot \vec{e}_l}_{\delta_{jl} = \sum_{j} D_{ij} D_{kj} \]
$ \Rightarrow$  D bildet paarweise orthogonale Spaltenvektoren\\
\[ (3) \Rightarrow D \left| \begin{matrix}
\vec{e}_1^\prime \cdot \vec{e}_1 & \vec{e}_2^\prime \cdot \vec{e}_1 & \vec{e}_3^\prime \cdot \vec{e}_1\\
\vec{e}_1^\prime \cdot \vec{e}_2 & \vec{e}_2^\prime \cdot \vec{e}_2 & \vec{e}_3^\prime \cdot \vec{e}_3\\
\vec{e}_1^\prime \cdot \vec{e}_3 & \vec{e}_2^\prime \cdot \vec{e}_3 & \vec{e}_3^\prime \cdot \vec{e}_3
\end{matrix} \right|  \vec{e}_1^\prime \cdot (\vec{e_2^\prime \times e_3^\prime} \underbrace{= }_{\text{orthogonales Rechtssystem}} 1 \]
\[ (4) \delta_{ik} = \sum_{i} D_{ij} D_{kj} = \sum_{i} D_{ij} (D^T)_{jk} \]
\[ \Rightarrow D^{-1} = D^T \Rightarrow DD^T = D^TD = \mathds{1} \]
\[ \text{ Dist} " orthogonale Matrix  \] 
\subsection{Beispiel: zweidimensionale Drehungen}
\[ D(\varphi) = \left( \begin{matrix}
cos \varphi & sin \varphi\\ - sin \varphi & cos \varphi
\end{matrix} \right) \] 
\[ \left( \begin{matrix}
\vec{e}_1^\prime \\ \vec{e}_2^\prime 
\end{matrix} \right) = \left( \begin{matrix}
cos \varphi & sin \varphi \\ - sin \varphi & cos \varphi
\end{matrix} \right) \left( \begin{matrix}
\vec{e}_1 \\ \vec{e}_2
\end{matrix} \right)  \]
\[ D(\varphi_2) D(\varphi_1) = D(\varphi_1) D(\varphi_2)  = D(\varphi_1 + \varphi_2) \]
für $\leq 3$ Dimensionen kommutieren Drehungen i.a. nicht!\\
\textbf{Transformation von Vektorzenpunkten: }
\[ \vec{r} = \sum_{i} x_i \vec{e}_i  = \sum_{i} x_i^\prime \vec{e}_i^\prime \]
\[ \Rightarrow x_j^\prime = \vec{e}_j^\orime \cdot \vec{r} = \sum_{i} x_i \vec{e}_j ^\prime \cdot \vec{e}_i = \sum_{i} D_{ji} x_i \]
also $ \vec{x}^\prime = D \vec{x} \Rightarrow$ Komponenten transformieren wie Einheitsvektoren\\
\textbf{Ausblick:}\\
 Bei nicht-orthogonalen Transformationen ist dies nicht der Fall und man unterscheidet ko. und kontravariante Vekoren.\\
 \textbf{Anwendung:}\\
$ |\vec{x}^\prime |^2 = \vec{x}^\prime \cdot \vec{x}^\prime = \vec{x}^{\prime T } \vec{x}^\prime = \vec{x}^T \underbrace{D^T D}_{\mathds{1 }   \vec{x} = \vec{x}^R \vec{x} = \vec{x} \cdot \vec{x} = |\vec{x}|^2 $ \\
	$ \Rightarrow $ Norm erhalten
\section{Lineare Differentialgleichungen}
Betrachte die gewöhnliche Differentialgleichung
\[ \ddot{x} + a(t) \dot{x} + b(t) x = f(t) \qquad (1)  \]
mit zeitabhängigen Koeffizenten. Wichtig z.B. zur Beschreibung von Schwingungen. \\
Gleichung 2.Ordnung kann in 2 Gleichungen 1.Ordnung umgeschrieben werden$_$
\[  \dot{x} = v \]
\[ \dot{v} = f(t) - b(t) x - a (t) v  \]
beschreibt z.B- Trajektorien im Phasenraum\\
$Gleichung (1)$ heißt \textit{inhomogene} wenn $ f(t) \neq 0$, ansonsten \textit{homogen}\\
Die Lösungen von homogenen Differentialgleichungen bilden einen linearen Vektorraum, d.h.Linearkombinationen von Lösungen sind wieder Lösungen, Fernen ist die Lösung für jede Anfangsbedingung $ x(t_0) = x_0, \dot{x} (t_0) = v_o  $ eindeutig.\\
\textbf{Satz}\\
\[ \text{Seien } x_1(t),x_2(t) \text{ zwei Lösungen der homogenen Differentialgleichung } \ddot{x} + a(t) \dot{x} + b(t) x = 0 \]
So erfüllt die Wronstzi-Determinante
\[ w(t) = x_1(t) \dot{x}_2 (t) - \dot{x}_1 (t) x_2 (t)   \]
die lineare Differentialgleichung 1. Ordnung
\[ \dot{w} (t) = -a(t) w(t) \]
\textbf{Beweis:}\\
\[ \dot{w} = x_1 \ddot{x}_2 - \ddot{x}_1 x_2 \underbrace{= }_{\text{Einfache Difgleichung.}} x_1(-a \dot{x}_2 - bx_2) - ( -a\dot{x}_1 - b x_1 )x_2  \]
\[ = - a (x_1 \dot{x}_2 - \dot{x}_1 x_2)  = - a W \] 
\textbf{Lösung dieser Gleichung:}
\[ \frac{dw}{w} = - a(t) dt \quad \Rightarrow ln \frac{w(t)}{w(t_0)} = - \int_{t_0}^{t} dt^\prime a (t^\prime) \]
$ \Rightarrow w(t) = w(t_0) e^{- \int_{t_0}^{t} dt^\prime a (t^\prime)} $\\
\textbf{Satz:}\\
Zwei Lösungen $x_1(t), x_2 (t)$ sind genau dann linear abhängig, wenn $ w(t) = x_1 (t)  \dot{x}_2 (t) - \dot{x}_1 x_2 (t) \equiv 0 $ \\
\textbf{Beweis:}
\[ \text{Sei } \lambda_1 x_1 + \lambda_2 x_2 \equiv 0 \text{ mit } \lambda_1 \neq 0 \]
\[ \Rightarrow x_1 = \frac{\lambda_2}{\lambda_1} x_2 \quad ; \quad \dot{x}_1 = \frac{\lambda_2}{\lambda_1} \dot{x}_2 \]
\[ \Rightarrow w = - \frac{\lambda_2}{\lambda_1}x_2 \dot{x}_2 + \frac{\lambda_2}{\lambda_1} \dot{x}_2 x_2 = 0  \]
für $ \lambda_2 \neq 0 $ analog\\
Sei $ w(t) \equiv 0 $. Wenn auf einem Integral $ x_1 , x_2 \neq 0 $
\[ \frac{\dot{x}_1}{x_1} = \frac{\dot{x}_2}{x_2} \Rightarrow \frac{dx_1}{x_1} = \frac{dx_2}{x_2} \Rightarrow ln x_1 (t) = ln x_2 (t) + const \]
\[ \Rightarrow x_1 (t) = const \cdot x_2 (t) \Rightarrow x_1 , x_2 \text{ linear abhängig} \]
\textbf{Satz:}\\
Sei $ x(t)$ eine auf einem Integral nicht verschiedene Lösung der homogenen Differentialgleichung, Dann ist
\[ y(t) = x(t) \int^{t} \frac{w(t^\prime)}{x^2(t^\prime) } dt^\prime \qquad ≥\text{mit} w(t) = e ^{\int^{t} a(t^\prime)dt^\prime}   \]
eine linear unabhängige Lösung, wobei die unteren Integrationsgrenzen beliebig sind.\\
\textbf{Beweis:}
\[ \dot{y} = \dot{x}  \]

\end{document} 