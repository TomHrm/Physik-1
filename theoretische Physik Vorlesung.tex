
\documentclass[11pt]{article}

%Menge: \mathbb{R}

\usepackage[ngerman]{babel}
\usepackage{amsmath} %align, für = untereinander einfach &=
\usepackage{amssymb}
\usepackage{amsthm}
\usepackage{listings}
\usepackage[utf8]{inputenc}
\usepackage{graphicx}
\usepackage{esvect}
\graphicspath{ {./images/} }
\usepackage{tikz}         % For arrow and dots in \xvec
\usepackage[left=2.30cm, right=2.30cm, top=1.70cm, bottom=2.00cm]{geometry}

% --- Macro \xvec
\makeatletter
\newlength\xvec@height%
\newlength\xvec@depth%
\newlength\xvec@width%
\newcommand{\xvec}[2][]{%
	\ifmmode%
	\settoheight{\xvec@height}{$#2$}%
	\settodepth{\xvec@depth}{$#2$}%
	\settowidth{\xvec@width}{$#2$}%
	\else%
	\settoheight{\xvec@height}{#2}%
	\settodepth{\xvec@depth}{#2}%
	\settowidth{\xvec@width}{#2}%
	\fi%
	\def\xvec@arg{#1}%
	\def\xvec@dd{:}%
	\def\xvec@d{.}%
	\raisebox{.2ex}{\raisebox{\xvec@height}{\rlap{%
				\kern.05em%  (Because left edge of drawing is at .05em)
				\begin{tikzpicture}[scale=1]
				\pgfsetroundcap
				\draw (.05em,0)--(\xvec@width-.05em,0);
				\draw (\xvec@width-.05em,0)--(\xvec@width-.15em, .075em);
				\draw (\xvec@width-.05em,0)--(\xvec@width-.15em,-.075em);
				\ifx\xvec@arg\xvec@d%
				\fill(\xvec@width*.45,.5ex) circle (.5pt);%
				\else\ifx\xvec@arg\xvec@dd%
				\fill(\xvec@width*.30,.5ex) circle (.5pt);%
				\fill(\xvec@width*.65,.5ex) circle (.5pt);%
				\fi\fi%
				\end{tikzpicture}%
	}}}%
	#2%
}
\makeatother

\usepackage{blindtext}
%\usepackage{bookman}

% --- Override \vec with an invocation of \xvec.
\let\stdvec\vec
\renewcommand{\vec}[1]{\xvec[]{#1}}
% --- Define \dvec and \ddvec for dotted and double-dotted vectors.
\newcommand{\dvec}[1]{\xvec[.]{#1}}
\newcommand{\ddvec}[1]{\xvec[:]{#1}}

%eigene Befehle:
\newcommand{\R}{\mathbb{R}}
\newcommand{\nablavec}{\vec{\nabla}}

\title{Theoretische Physik 1}

\date{\today}
\begin{document}
\lstset{language=Java}
\author{Tom Herrmann}
\linespread{2.05}
\maketitle
\section{Einleitung}
	Gliederung:
	\begin{itemize}
		\item \textbf{Mathematische Grundlagen:}\\Vektoren, krummlinige Koordinatensysteme, Differential- und Integralrechnung, partielle Ableitungen, Vektoranalysis, Gradient, Diverenz, Rotation, lineare gewöhnliche Differentialgleichungen, Matrizen und Tensoren, Fourier-Transformation, komplexe Zahlen, Wahrscheinlichkeit.
		\item \textbf{Grundlagen der Mechanik:} Kinematik und Dynamik von Massenpunktsystemen, Newtonsche Axiome, Arbeit, konservative Kräfte, Schwingungen, Zentralfeld, Kepler-Problem, beschleunigte Bezugssysteme
		\item \textbf{Spezielle Relativitätstheorie:}  Lorentz-Transformation und kinematische Konsequenzen, Minkowski-Raum, Vierer-Impuls, relativistische Bewegungsgleichung, Energie-Impuls-Vektor, Äquivalenz von Masse und Energie
		\item \textbf{Wärmelehre: } Boltzmann Verteilung, Entropie, Irreversibilität
	\end{itemize}
		\textbf{Wenn man eine Fettgedruckte Größe in den Übungsaufgaben sieht ist damit ein Vektor gemeint}
\part{Vorlesung 1}
	$\#$ = Anzahl
	\section{Vektorrechnung}
		Vektoren sind gerichtete Größen deren Komponenten bei Drehung oder allgemeiner Koordinaten Transformationen gewisse transformationseigenschaften besitzen. \\
		Skalare sind dabei invariant unter Koordinatentransformation. (z.B: Masse, Ladung, Temperatur)\\
		Ortsvektor $\vec{r}$, Geschwindigkeit $\vec{v} = \frac{d\vec{r}}{dt} = \frac{d^2\vec{r}}{dt^2} $\\
		Vektoren haben eine Richtung und eine Länge.
	\section{Ableitung}
		Ableitung können sowas als $f`$ geschrieben werden als auch als $\dot{f}$ somit kann die Geschwindigkeit $\vec{v}$ als die Ableitung der Position ausgedrückt werden $\vec{v}$ = $\vec{\dot{x}}$.
	\section{Drehung}
	$\vec{a} = (a_x, a_y) = (x, y)$\\
	$\vec{a} = (a_u, a_v) = (u, v)$ dabei ist für u und v das Koordinatensystem einfach nur um einen bestimmten Winken $\varphi$ gedreht.\\
	\[u = x cos \varphi + y sin \varphi\]
	\[v = - x sin \varphi + y cos \varphi\]
	Länge von: \[\vec{a} = \sqrt{u^2+ v^2} = [\left(x cos \varphi + y sin \varphi)^2 (- x sin \varphi + y cos \varphi)^2\right]^\frac{1}{2}\]
	\[	[x^2 cos^2 \varphi + y^2 sin^2\varphi+xy cis \varphi sin \varphi + x^2 sin^2 \varphi + y^2 cos^2 \varphi - 2xy sin \varphi cos \varphi]^\frac{1}{2}\]
	\[ [x^2(cos^2\varphi + sin^2 \varphi) + y^2 sin^2\varphi + cos^2 \varphi]^\frac{1}{2} \]
	\[ \sqrt{x^2 + y^2} \] \qed \\
Drehung als Matrix Multiplikation: $\left( \begin{array}{c}
	u \\ 
	v
\end{array} \right)
= \left( \begin{array}{cc}
	cos \varphi & sin \varphi \\ 
	-sin \varphi & cos \varphi
\end{array} \right)
  \left( \begin{array}{ccc}
 	x \\ 
 	y
 \end{array} \right)
= \frac{2}{2} = A_{ij} x_j $  

\part{Vorlesung 2}
	\section{Basis der Vektorrechnung}
	Mathematisch abstrakt ist ein linearer oder Vektorraum ein "Körper" von Elementen $\vec{a}, \vec{b}, \vec{c}$, im dem eine Addition und eine ultiplikation mit skalaren $\alpha$ definiert ist.
		
		\subsection{Basis ausrechnung}
			\begin{itemize}
				\item Die Anzahl der Vektoren stimmt überein mit der Dimension des Vektorraumes.
				\item Die Vektoren sind linear unabhängig.
			\end{itemize}
		\subsection{Addition}
		\[	\left(\begin{array}{c}
				1\\ 
				2
			\end{array} \right) + \left(\begin{array}{c}
			3\\ 
			5
		\end{array} \right) =	\left(\begin{array}{c}
		1 + 3\\ 
		2 + 5
	\end{array} \right) = \left(\begin{array}{c}
	4\\ 
	7
\end{array} \right) \]
		\subsection{Subtraktion}
		\[	\left(\begin{array}{c}
				3\\ 
				4
			\end{array} \right) - \left(\begin{array}{c}
			2\\ 
			5
		\end{array} \right) =
	\left(\begin{array}{c}
		3 - 2\\ 
		4 - 5
	\end{array} \right) = \left(\begin{array}{c}
	1\\ 
	-1
\end{array} \right) \]
		\subsection{Multiplikation}
		\[	a * \vec{v} = 5 * \left(\begin{array}{c}
				2\\ 
				1\\
				3
			\end{array} \right) = \left(\begin{array}{c}
			5 * 2 \\ 
			5 * 1 \\
			5 * 3
			\end{array} \right) =
			\left(\begin{array}{c}
			10\\ 
			5\\
			15
			\end{array} \right) 	\]  
			Bei der Division läuft das ganze dann fast genau so ab einfach nur 5 in den als zähler der jeweiligen Koordinate schreiben.
		\subsection{Skalarprodukt}
			\[ \vec{a} \circ \vec{b} = |\vec{a}| \circ |\vec{b}| = \begin{pmatrix} a_1 \\ a_2 \\ a_3 \end{pmatrix} \circ \begin{pmatrix} b_1 \\ b_2 \\ b_3 \end{pmatrix} = a_1 \cdot b_1 + a_2 \cdot b_2 + a_3 \cdot b_3 \]
		\section{Eigenschaften von Rechenoperationen}
		Kommutativität $\vec{a} + \vec{b} = \vec{b} + \vec{a}$\\
		Distributivität $\lambda * (\vec{a} + \vec{b}) = (\vec{a} + \vec{b}) * \lambda$\\
		Homogenität (gemischtes Assoziativgesetz) $\alpha(\vec{a} + \vec{b} = (\alpha \vec{a}) * \vec{b} =  \vec{a} * (\alpha \vec{b})$
		
		
		\subsection{Schwarzsche Ungleichung}
			$|\vec{a} * \vec{b}| \leq |\vec{a}| |\vec{b}|$
			
		\subsection{Projektion auf die Richtung \vec{b}}
			$a_b := a cos(\varphi) = \vec{a} * \vec{b} \text{ wo } \vec{b} := \frac{\vec{b}}{|\vec{b}|}$ der Einheutsvektor in $\vec{b}$ Richtung ist. (Also soll hier b der Einheitsvektor sein)\\
			Abstrake Definition eines Skalarproduktes mit obigen Eigenschaften und zusätzlichen $\vec{a}*\vec{a} > 0   \exists \vec{a} \neq 0 $ (positiv definiert)\\
		\part{Vorlesung 3}
			\subsection{Vektorprodukt}
				$\vec{c} = \vec{a} \times \vec{b}$ \qquad $|\vec{c}| = c = ab$  \;  $ |sin \varphi|$
		\begin{center}
			\includegraphics[scale=0.3]{vektorprodukt.png}
		\end{center}
	$\Leftarrow$ c = Fläche des von $\vec{a}$ und $\vec{b}$ aufgespannten Paralellogramms Richtung bestimmt durch \textbf{Rechtsschrauben regel}: Drehe $\vec{a}$ in Richtung $\vec{b}$
		\begin{center}
		\includegraphics[scale=0.3]{vektorprodukt2.png}
	\end{center}
	\textbf{Eigenschaften:}\\
	Antikommutativität: Das heißt, bei Vertauschung der Vektoren wechselt es das Vorzeichen\\ $\vec{a}\times\vec{b} = -\, \vec{b}\times\vec{a}$\\	
	Distributivität	$\vec{a} \times (\vec{b} + \vec{c} = \vec{a} \times \vec{b} + \vec{a} \times \vec{c})$\\
	Homogenität (gemischtes Assoziativgesetz) $\alpha(\vec{a} + \vec{b} = (\alpha \vec{a}) * \vec{b} =  \vec{a} * (\alpha \vec{b})$\\
	\textbf{Sonderfälle:}\\
	$ b(\vec{a}\times\vec{b}) \times (\vec{b}\times\vec{c})  = \vec{b} \cdot \det(\vec{a},\vec{b},\vec{c}) $\\
	$ (\vec{a}\times\vec{b}) \times (\vec{a}\times\vec{c})  = \vec{a} \cdot \det(\vec{a},\vec{b},\vec{c}) $\\
	$ (\vec{a}\times\vec{b}) \times (\vec{a}\times\vec{b})  = \vec{0}$
	\subsection{Komonentendarstellung}
		Definiere 3 \"orthogonale\" das heißt orthogonale Einheitsvektoren\\
			$\vec{\hat{a}} , \vec{\hat{y}}, \vec{\hat{z}}$ \\der 
			im 3-Dimensionalen euklidischen Raum, mit (Schreibregel: Der einfachheit wegen lasse ich in den 2 Zeilen das Zeichen für Einheitsvektor weg, es sollte eigentlich dabei stehen) \\
			$\vec{x} * \vec{x} = \vec{y} * \vec{y} = \vec{z} * \vec{z} = 1   ; \vec{x} * \vec{y} = \vec{x} * \vec{z} = \vec{y} * \vec{z}  = 0$\\
			und\\
			$\vec{x} \times \vec{y} = \vec{z} , \vec{y} \times \vec{z} = \vec{x}, \vec{z} \times \vec{x} = \vec{y}$\\
			\textit{(Ab diesem Zeitpunkt ist die Schreibregel wieder aufgehoben)} andere häufige Schreibweisen:\\
				$\vec{e}_x, \vec{a}_y, \vec{a}_z ; \vec{e}_1, \vec{e}_2, \vec{e}_3$
				\begin{center}
					\includegraphics[scale=0.45]{komponentendarstellung.png}
				\end{center}
			$\vec{a} = a_x \vec{\hat{x}} + a_y \vec{\hat{y}} + a_z \vec{\hat{z}} = (a_x, a_y, a_z)$ mit $a_x = \vec{a} * \vec{\hat{x}}, a_y = \vec{a} * \vec{\hat{y}}, a_z = \vec{a} * \vec{\hat{z}}$
				\begin{center}
					I	\includegraphics[scale=0.4]{Komponentendarstellung2.png}
				\end{center}
\subsection{Spatprodukt}
\begin{center}
	\includegraphics[scale=0.3]{1000px-Parallelepiped2.png}
\end{center}
Das Spatprodukt (\vec a, \vec b, \vec c) dreier Vektoren $\vec{a}, \vec{b}$ und $\vec{c}$ des dreidimensionalen euklidischen Vektorraums \R$^3$ kann wie folgt definiert werden: \\
	$(\vec{a},\vec{b},\vec{c}) = (\vec{a} \times \vec{b}) \cdot \vec{c}$
		\subsubsection{Eigenschaften des Spatprodukts}
			\begin{itemize}
				\item Das Spatprodukt ist \textbf{nicht kommutativ}. Der Wert ändert sich jedoch nicht, wenn man die Faktoren zyklisch vertauscht:\\
				$(\vec{a} \times \vec{b}) \cdot \vec{c} = (\vec{b} \times \vec{c}) \cdot \vec{a} = (\vec{c} \times \vec{a}) \cdot \vec{b}$
				\item Man kann das \textbf{Spatprodukt} \textbf{mit Hilfe der Determinante berechnen}. Für\\ $\vec a = \begin{pmatrix} a_1 \\ a_2 \\ a_3 \end{pmatrix},\ \vec b = \begin{pmatrix} b_1 \\ b_2 \\ b_3 \end{pmatrix},\ \vec c = \begin{pmatrix} c_1 \\ c_2 \\ c_3 \end{pmatrix}$ gilt:\\ $(\vec{a},\vec{b},\vec{c}) = \det\begin{pmatrix} a_1 & b_1 & c_1 \\ a_2 & b_2 & c_ 2 \\ a_3 & b_3 & c_3\end{pmatrix} =
				\det\begin{pmatrix} a_1 & a_2 & a_3 \\ b_1 & b_2 & b_ 3 \\ c_1 & c_2 & c_3 \end{pmatrix}$
				\item Die \textbf{Multiplikation} mit einem Skalar $\alpha \in \mathbb{R}$ ist \textbf{assoziativ }\\
					$( \alpha \cdot \vec{a}, \vec{b}, \vec{c} ) = \alpha \cdot ( \vec{a}, \vec{b}, \vec{c} )$
				\item Es gilt ein \textbf{Distributivgesetz:}\\
				$( \vec{a}, \vec{b}, \vec{c} + \vec{d} ) = ( \vec{a}, \vec{b}, \vec{c} ) + ( \vec{a}, \vec{b}, \vec{d} )$
				\item Invarianz unter zyklischer Vertauschung: \\
				$(\vec{a} \times \vec{b}) * \vec{c} = (\vec{b} \times \vec{c}) * \vec{a} = (\vec{c} \times \vec{a} * \vec{b})$\\
				Dies ist auch in 3-Dimensionen möglich, wie man am Beispiel der Determinante gesehen hat.
			\end{itemize}
		\subsection{Doppeltes Vektorprodukt}
			Im algemeinen nicht assoziativ\\
			Es gibt die \textbf{bac-cab-Regel}\\
			$\vec{a} \times (\vec{b} \times \vec{c})= \vec{b} (\vec{a} * \vec{c}) - \vec{c}(\vec{a} *\vec{b})$\\
			Daraus folgt auch die  Jacobi-Identität \\
			 	$\vec{a}\times(\vec{b}\times\vec{c}) +\vec{b}\times (\vec{c}\times\vec{a}) +\vec{c}\times (\vec{a}\times\vec{b}) = 0$
	\section{Das Differential}
		Ist $f(x)$ differenzierbar bei x, so nennt man $f^\prime(x) h$ für beliebige h  Differential  von $f(x)$. Ma schreibt oft
			\begin{center}
				$dy = df(x) = f^\prime(x)dx$
			\end{center}
			In der Mathematik wird dies Differentialform genannt, die man formal als dual zum Vektorraum bestehend aus Koordinaten $x, ...$ definiert.
	\section{Vektorfunktionen}
		z.B. Kurve eines Massenpunktes wird beschrieben durch $\vec{r}|t| = (x(t), y(t), z(t))$ wobei t die Zeit ist.\\
				\begin{figure}[htbp]
					\begin{minipage}[t]{10cm}
						\vspace{0pt}
						\centering
						\includegraphics[scale=0.5]{Vektorfunktionen.png}
					\end{minipage}
					\hfill
					\begin{minipage}[t]{10cm}
						\vspace{0pt}
						$\vec{\dot{r}} = \frac{d\vec{r}}{dt}\\ = \lim_{\Delta t\to\ 0} \frac{\vec{r}(t + \Delta t) - \vec{r}(t)}{\Delta t} \\= (\frac{dx}{dt}, \frac{dy}{dt}, \frac{dz}{dt}) =$ Geschwindigkeit $\vec{v}$\\
						Beschleunigung = $\vec{a} = \frac{d\vec{v}}{dt} = \frac{d^2\vec{r}}{dt^2}$\\
					\end{minipage}
				\end{figure}
				\textbf{Kettenregel, Produktregel etc. gelten auf für Vektorfunktionen.}\\
				z.B.\\
				\[\frac{d}{dt}(\vec{a} \times \vec{b}) = \frac{d\vec{a}}{dt} \times \vec{b} \vec{a} \times \frac{d\vec{b}}{dt}\]
				\[ \frac{d}{dt}\vec{r} (f(t^\prime)) = \frac{d\vec{r}}{dt}(f(t^\prime)) * \frac{df}{dt^\prime}(t^\prime) \]
	\section{Integration}
	\includegraphics[scale=0.05]{integral.png}
	Die Integralrechnung ist aus dem Problem der Flächen- und Volumenberechnung entstanden.
	Dabei kann die Funktion auch ein komplett random geformter klotz in einem koordinaten System sein.\\
	$\int_a^b f(x)\,\mathrm dx$\\
	\begin{figure}[htbp]
		\begin{minipage}[t]{6cm}
			\vspace{0pt}
			\centering
			\includegraphics[scale=0.45]{Integrationbeispiel.png}
		\end{minipage}
		\hfill
		\begin{minipage}[t]{6cm}
			\vspace{0pt}
			Eine der am  häufigsten genutzen Regeln in der Physik ist, dass Konstanten vor das Integral gezogen werden können.
		\end{minipage}
	\end{figure}
\section{Differentialrechnung}
	\begin{center}
		$dy = df(x) = f^\prime(x)dx$
	\end{center}
	In der Mathematik wird dies Differentialform genannt, die man formal als dual zum Vektorraum bestehend aus Koordinaten $x, ...$ definiert.
	\newpage
	\section{Vektorfunktionen}
	z.B. Kurve eines Massenpunktes wird beschrieben durch $\vec{r}|t| = (x(t), y(t), z(t))$ wobei t die Zeit ist.\\
	\begin{figure}[htbp]
		\begin{minipage}[t]{10cm}
			\vspace{0pt}
			\centering
			\includegraphics[scale=0.4]{Differentialrechnung.png}
		\end{minipage}
		\hfill
		\begin{minipage}[t]{10cm}
			\vspace{0pt}
			$f^\prime(x_0) = \frac{df}{dx} |_{x_0}$$ \\ = lim_{\Delta x \to\ 0} \frac{\Delta f}{\Delta x}\\ = lim_{x \to\ x_0} \frac{f(x) - f(x_0)}{x - x_0}$
		\end{minipage}
	\end{figure}
\section{Taylor Entwicklung}
	Im 1. Ordnung: $f(x) \approx f(x_0) + f^\prime(x_0)(x-x_0)$\\
\textbf{Verallgemeinerung}:\
	\[ f(x)= \sum_{n=0}^{\infty}\frac{f^(^n^) (x_0)}{n!} (x-x_0)^n = f(x_0) + f^\prime (x_0) (x - x_0) + \frac{f^\prime ^\prime (x_0)}{2} (x-x_0)^2 + ... \]
	Jede \textbf{Potenzreihe} $\sum_{n=0}^{\infty}a_n(x-x_0)^n$ und damit auch jede \textbf{Taylor-Reihe} hat einen Konvergenzradius $\lambda$, so daß $\sum_{n=0}^{\infty}a_n(x-x_0)^n$ für alle $|x-x_0| < r$ konvergiert. Im allgemeinen ist r durch den Abstand zur nöchsten Singularität von $f(x)$ gegeben, diese Tatsache gilt aber nur allgemein im Raum der komplexen Zahlen.\\
	\textbf{Beispiel}:\\ 
	$f(x) = \frac{1}{1 + x^2} = \sum_{n=0}^{\infty} (-1)^n x^2^n = \sum_{n=0}^{\infty} a_nx^n $ hat den Konvergenzradius $r = 1$, da $\frac{a_2_n_+_2}{a_2_n} = -x^2$ obwohl $\frac{1}{1+x^2}$ für alle $x \in \R$ wohldefiniert und unendlich oft differenzierbar ist. \\
	Der tiefere Grund ist daß $ \frac{1 }{1 + x^2} $ bei $x = \pm i$ singulär ist. 
	\section{Partielle Differentiation}
		Funktionen mehrerer Variablen, z.B. f(x,y,z), können nach den einzelnen Variablen differenziert werden, wobei man sich die anderen Variablen konstant vorstellt.\\
		\[ \frac{\partial f}{\partial x}(x,y,z) =  lim_{\Delta x \to\ 0} \frac{f(x+\Delta x,y,z - f(x,y,z))}{x,y,z} \]
		und analog für die anderen Variablen.\\ \textbf{Kurzschreibweise}:\\
		\[ f_x = \frac{\partial f}{\partial x} \quad f_y = \frac{\partial f}{\partial y} \quad f_z = \frac{\partial f}{\partial z} \] \\
		\textbf{Höhere Ableitungen}:\\
		\[  f_x_x = \frac{\partial}{\partial x}(\frac{\partial f}{\partial x}) = \frac{\partial^2f}{\partial x^2} \]
	\[  f_x_y = \frac{\partial}{\partial y} (\frac{\partial f}{\partial x}) = \frac{\partial^2 f}{\partial_y \partial_x} \]
		 übrigens gilt: $\frac{\partial^2 f}{\partial y \partial x } = \frac{\partial^2 f}{\partial_x \partial_y}$ (Symmetrie)\\
		 Man kann leicht die erweiterte Kettenregel zeigen:\\
		 Für $f(t) = f(x|t) , y(t) , z(t)$ gilt $\frac{df}{dt} = \frac{\partial f}{\partial x} \frac{dx}{dt} frac{\partial f}{\partial y} \frac{dy}{dt} + frac{\partial f}{\partial z} \frac{dz}{dt}$\\ oder als Differential \\
		 $df = \frac{\partial f}{\partial x} dx + \frac{\partial f}{\partial y} dx + \frac{\partial f}{\partial z} dz   $\\
		 Kann auch für die 1. Ordnung einer multi-dimensionalen Taylor-Entwicklung verwendet werden: \\
		 $ f(x,y,z) = f(x_0, y_0, z_0) + \frac{\partial f}{\partial x} (x_0, y_0, t_0)(x-x_0) + \frac{\partial f}{\partial y} (x_0, y_0, t_0)(y-y_0) + \frac{\partial f}{\partial z} (x_0, y_0, t_0)(z-z_0) + ... $
	
		\section{Vektoranalysis: Gradient, Divergenz, Rotation}
			Ein \textbf{Vektorfeld} ist eine Vektor-wertige Funktion mehrerer Variablen:
			\[ \vec{F}(\vec{r}) = (F_x(x,y,z), F_y(x,y,z), F_z(x,y,z) ) \]
			\textbf{Beispiele:} Geschwindkeitsfeld (e.g. Windgeschwindigkeit), Kraftfeld\\
			Ein \textbf{Skalarfeld} hat nur eine (skalare) Komponente:
				\[ f(\vec{r}) = f(x,y,z) \]\\
				\textbf{Gradient:}\\
				
				\[ f \rightarrow \vec{\nabla} f = grad f = \vec{\nabla} f = \left( \frac{\partial f}{\partial x}, \frac{\partial f}{\partial y}, \frac{\partial f}{\partial z} \right) \text{\qquad} Skalarfeld \rightarrow Vektorfeld \] 
				Es gelten die üblichen Produkt- und Summenregeln:
					\[ \vec{\nabla}(f+g) = \vec{\nabla }f + \vec{\nabla}g; \vec{\nabla}(f+g)= g\vec{\nabla}f + f\vec{\nabla}g \]
				\textbf{Beispiel:} f sei Funktion der Abstand $|\vec{r} - \vec{r}_0| = f^\prime (| \vec{r}-\vec{r}_0|) \frac{x-x_0}{| \vec{r} - \vec{r}_0 |}$ \\
				\textit{Dabei gilt:} $|\vec{r} - \vec{r}_0| = \sqrt{(x-x_0)^2 + (y-y_0)^2 + (z-z_0)^2}$\\
				Dabei ist das ganze dann logischer weise nur vom Abstand abhängig\\
				$\Rightarrow \vec{\nabla} (\vec{r}) =   \left( \frac{\partial f}{\partial x}, \frac{\partial f}{\partial y}, \frac{\partial f}{\partial z} \right) = \frac{1}{|\vec{r} - \vec{r}_0|} ( x-x_0, y- y_0, z- z_0 )    \frac{\vec{r} - \vec{r}_0}{| \vec{r}-\vec{r}_0 |} = \vec{e}_{\vec{r} - \vec{r}_0}$ \\
				\[ \frac{\partial r}{\partial x} = \frac{1}{2} \frac{2(x-x_0)}{\sqrt{(x-x_0)^2 + (y-y_0)^2 + (z-z_0)^2}} = \frac{x-x_o}{|\vec{r - \vec{r}_0}|} \]
				Einheiten in Kegelkoordinaten:\\
				$\Rightarrow \vec{\nabla} = \frac{df}{dr}(| \vec{r}-\vec{r}_0 |)  \vec{\nabla}| \vec{r}-\vec{r}_0 | = f^\partial (| \vec{r}-\vec{r}_0 |) \frac{\vec{r} - \vec{r}_0}{| \vec{r}-\vec{r}_0 |}$
				z.B. $ f(\vec{r}) = c | \vec{r}-\vec{r}_0 |  $ Das Gravitationspotential zwischen zwei Teilchen der Masse $m_1$ und $m_2$\\
				 $ \phi(\vec{r}_1 , \vec{r}_2) = -\frac{G_N m_1 m_2}{| \vec{r}-\vec{r}_0 |}$ $\alpha = -1$ und $c = -G_N m_1 m_2$\\
				 
				 $\Rightarrow  \vec{\nabla} \phi (| \vec{r}-\vec{r}_0 |) = \frac{G_N m_1 m_2}{| \vec{r}-\vec{r}_0 |} \frac{\vec{r} - \vec{r}_0}{| \vec{r}-\vec{r}_0 |} $ 
				 Nebenrechnung: $\phi^\partial (r) = + \frac{G_N m_1 m_2}{r^2}\\
				 \vec{F}_1_2 = - \vec{\nabla} \phi = - \frac{G_N m_1 m_2}{| \vec{r}-\vec{r}_0 |}(\vec{r} - \vec{r}_0)$ = Die Kraft die Teilchen 2 auf Teilchen 2 ausübt.
				
				\section{Rotation}
				Hierbei muss extrem auf die Vorzeichen aufgepasst werden. Es gibt zudem nur 2 Kombinationen wie man im laufe sehen wird die sich nicht wegkürzen.\\
					rot($\vec{F} (\vec{r}) = \vec{\nabla} \times \vec{F} = \sum_{i, j, k}^{}$ $\frac{\partial f_j}{\partial x_i}  \vec{e}_{i,j} = (\frac{\partial f_z}{\partial y} - \frac{\partial f_y}{\partial z}, \frac{\partial f_x}{\partial z} - \frac{\partial f_z}{\partial x}, \frac{\partial f_z}{\partial x} - \frac{\partial f_x}{\partial y}$)\\
					Es sind 3 da wir uns im 3 Dimensionalen befinden.  \quad  Vektorfeld $\Rightarrow$ Vektorfeld ; Wirbelstärke\\
					
					\textbf{Summe und Rotation:}\\
						$\vec{\nabla} \times(\vec{F} + \vec{G}) = \vec{\nabla} \times \vec{F} + \vec{\nabla} \times \vec{G}$\\
						$ \vec{\nabla} \times(f \vec{F} )= f \vec{\nabla} \times \vec{F} (\vec{\nabla} f) \times \vec{F} $\\
						\textbf{Beispiele:}
							$\vec{f} (\vec{r}) = \vec{F} = f(r) \vec{\nabla} \times \vec{r} + (\vec{\nabla} f) \times \vec{r}$\\
							$\vec{f} (\vec{r}) = (\frac{\partial f_z}{\partial y} - \frac{\partial f_y}{\partial z}, \frac{\partial f_x}{\partial z} - \frac{\partial f_z}{\partial x}, \frac{\partial f_z}{\partial x} - \frac{\partial f_x}{\partial y)} = \vec{0}  $ \qquad $\vec{r} = (x,y,z)$\\
							$\vec{\nabla}f = f^\prime \frac{\vec{r}}{|\vec{r}|} \Rightarrow (\vec{\nabla}f) \times \vec{r} = \frac{f^\prime}{|\vec{r}|} \times \vec{r} = 0$\\
							
							\textbf{Beispiel B:}\\
							$\vec{F}(\vec{r}) = \vec{w} \times \vec{r}$\\
							$| \vec{F}(\vec{r})| = |\vec{w}| \sqrt{x^2+y^2}$\\
							$(\vec{\nabla}  \times \vec{F})_x = \frac{\partial f_z}{\partial y} - \frac{\partial f_y }{\partial z}  = \frac{\partial}{\partial y} (w_x y - w_y x) - \frac{\partial}{\partial z}(w_y x - w_x y)$ \\
							Damit gilt: $\vec{F} = \vec{w} \times \vec{a} \Rightarrow F_z = w_xy - w_yx$ und $F_y = w_zx - w_xz$\\
							Also ist es am ende: $\Rightarrow \vec{\nabla} \times \vec{F} = 2\vec{w}$
							
							$\Leftrightarrow \vec{\nabla} \times (\vec{w} \times \vec{r})= 2\vec{w}$
							
					\section{Divergenz}
						div $\vec{f} = \vec{\nabla}, \vec{F} = (\frac{\partial}{\partial x}, \frac{\partial }{\partial y} , \frac{\partial }{\partial z}) * F_x , F_y, F_z = \frac{\partial F_x}{\partial x} + \frac{\partial F_y}{\partial y}+ \frac{\partial F_z}{\partial z}$\\
						Vektorfunktion $\rightarrow$ Sklarafunktion\\
						$\vec{\nabla } (\vec{F} + \vec{G}) = \vec{\nabla} \vec{F} + \vec{\nabla} \vec{G} $ \quad $ \vec{\nabla}+ (f \vec{f}) = f \vec{\nabla} * \vec{F} +(\vec{\nabla f}) * \vec{F} $\\
						\textbf{Beispiele:}\\
						$\vec{F} (\vec{r}) = \vec{r}$ \quad $\vec{\nabla} * \vec{r} = 3 $\\
						anderes Beispiel: $\vec{F} (\vec{r}) = \vec{w} \times \vec{r}$ \quad $\vec{w} = const.$\\
						wähle deine Einschränkungen: $\vec{w} = w\vec{e}_z = (0,0,w) \rightarrow \vec{w} \times \vec{r} = (-wy, wx, 0)$\\
						$\vec{\nabla} * (\vec{w} \times \vec{ r}) = \frac{\partial}{\partial x} (wx) + \frac{\partial}{\partial z} 0 = 0$\\
						Weitere Beispiele im im Skript.\\						
					
\end{document}